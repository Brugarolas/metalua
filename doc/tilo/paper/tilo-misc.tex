%-*-mode:latex; eval:(whizzytex-mode);-*-
\documentclass{article}
\usepackage{amssymb}
\title{A Typed Lua Calculus}
\author{Fabien Fleutot}
\newcommand{\Function}[2]{\texttt{function(}#1\texttt{)\ }#2\texttt{\  end}}
\newcommand{\Return}[1]{\texttt{return\ }#1}
\newcommand{\Local}[1]{\texttt{local\ }#1}
\newcommand{\Nil}{\texttt{nil}}
\newcommand{\Table}[1]{\{\ #1\ \}}
\newcommand{\eqlabel}[1]{(\textit{#1})}
\newcommand{\compset}[2]{\left\{ #1\ \middle/\ #2 \right\}}
\newcommand{\Leq}{\stackrel{L}{=}}
\begin{document}
\maketitle

\section{Type system}

\subsubsection{Syntax}

\paragraph{Expressions}
Expression types are either primitives (bool, number etc.) or
structural table types, inspired by the $\varsigma$-caluculus and
R\'emy records. Given the verbosity of the latter, a system of names
aliases will be needed.

Some libraries in default libraries require a type union
operator. Such an operator wouldbreak apart any hope of type
inference, so introducting it in the system will only be envisionned
latter.

In the type system we add a joker \# for dynamic terms, those which
will be typed at runtime. This type is neither top nor bottom: because
it must be accepted everywhere it's similar to bottom, and beacause
every type can be ``weakened'' into it it's similar to top, but
admitting that top=bottom would collapse the whole type system into
something unusable where all types are equal. Therefore we'll rely on
the concept of ``compatibility'', mostly orthogonal with subtyping,
which is the core of gradual typing.

\begin{equation}
\begin{array}{rcll}
T_E &::=& \texttt{nil} | \texttt{boolean}
     | \texttt{string} | \texttt{number} & \eqlabel{T-Prim} \\
    &|&   v & \eqlabel{T-Opaque} \\
    &|&   [\overline{E: T_F}; T_F] & \eqlabel{T-Table}\\
    &|&   T_{\bar E} \rightarrow T_{\bar E} & \eqlabel{T-Func} \\
    &|&   \# & \eqlabel{T-Dynamic} \\
\end{array}
\end{equation}

\paragraph{Expression sequences} For function arguments and returns, list
of values can be taken and returned. These lists might have a fixed
size (i.e. all elements of index big enough have type {\tt nil}), or
an unknown type \# indicating that we have no idea what might be
passed as extra arguments. So the last type listed by an expression
list type is the one of all remaining argument, up to $+\infty$.

\begin{equation}
\begin{array}{rcll}
T_{\bar E} &::=& \bar E; E... & \eqlabel{T-Sequence} \\
\end{array}
\end{equation}

\paragraph{Table fields}
Out of R\'emy's calculus, we retain the following field types in a
first step: Unknown ($U$), present ($P(T)$). The ``absent'' type is
also in the system, encoded as {\it P(Nil)}. Unknown means we don't
know whether this given field is present or not in the table, so we
can neither get it nor set it (because it might be already there with
an uncompatible type). Present(T) means the field is present and has
type T. It can be read with {\it(Index)}, and the resulting term
will have type T; it can also be written with {\it(Set)}, accepting
terms of types smaller than T.

In latter steps, we'll probably want to introduce {\it Virtual(T)}, a
field that's needed for the table to be callable, but still not
provided. A table with virtual fields can be written but not read;
writing in a virtual field turns it into a present one.

We'll also introduce the $\varsigma$ binder, which allows object types
to refer to themselves. This operator is mainly tricky in case of
inheritance, a construction which is not central to Lua (the binder
introduces contravariance in object types which are otherwise
covariant)

\begin{equation}
\begin{array}{rcll}
T_F &::=& T_E & \eqlabel{T-Present} \\
  &|&   ? & \eqlabel{T-Unknown} \\
\end{array}
\end{equation}

\paragraph{Statements}
Statement types describe what they return

\begin{equation}
\begin{array}{rcll}
T_S &::=& x & \eqlabel{X} \\
  &|&   y & \eqlabel{Y} \\
\end{array}
\end{equation}


\begin{equation}
\begin{array}{rcll}
T_{\bar S} &::=& x & \eqlabel{X} \\
  &|&   y & \eqlabel{Y} \\
\end{array}
\end{equation}

\subsection{Typing rules}

$$ \frac
  {\Gamma \vdash E_1: T_{\bar E 2} \rightarrow T_{\bar E}
    \quad
    \Gamma \vdash \bar E: T_{\bar E 2}}
  {\Gamma \vdash E_1(\bar E): T_{\bar E}}
  \quad
  (\textit{R-Apply})$$


\section{Type annotated terms}

\begin{equation}
\begin{array}{rcll}
E &::=& L & \eqlabel{Left} \\
  &|&   P & \eqlabel{Primitive} \\
  &|&   E(\bar E) & \eqlabel{Apply-E} \\
  &|&   \Function{\overline{v:T_E}; T_E...}{\bar S}:T_{\bar E} & \eqlabel{Function} \\
  &|&   \Table{ \overline{[E]=E} } & \eqlabel{Table} \\
\end{array}
\end{equation}

\begin{equation}
\begin{array}{rcll}
L &::=& v & \eqlabel{Variable} \\
  &|&   E[E] & \eqlabel{Index} \\
\end{array}
\end{equation}

\begin{equation}
\begin{array}{rcll}
P &::=& \langle\textrm{string}\rangle & \eqlabel{String} \\
  &|&   \langle\textrm{number}\rangle & \eqlabel{Number} \\
  &|&   \texttt{true} & \eqlabel{True} \\
  &|&   \texttt{false} & \eqlabel{False} \\
  &|&   \texttt{nil} & \eqlabel{Nil} \\
\end{array}
\end{equation}

\begin{equation}
\begin{array}{rcll}
S &::=& \Local{\bar v:\bar E} & \eqlabel{Local} \\
  &|&   \bar L = \bar E & \eqlabel{Set} \\
  &|&   E(\bar E) & \eqlabel{Apply-S} \\
  &|&   \Return \bar E & \eqlabel{Return} \\
\end{array}
\end{equation}

Type annotations are added at local var creations (function parameters
and local statements), and the functions' result sequences are
type-declared as well. In concrete syntax, lack of explicit
declaration will be interpreted as dynamic type \#.


\section{scratch}

Unmanaged: weird operators involving metatables and fenv, arithmetic
operators (especially overridden ones), metamethods, equality
operators.

\begin{equation}
\begin{array}{rcll}
X &::=& x & \eqlabel{X} \\
  &|&   y & \eqlabel{Y} \\
\end{array}
\end{equation}

``{\tt repeat S until E}'' is encoded as ``{\tt while true do S; if E then
  break end end}''.

``{\tt local v = E}'' is encoded as ``{\tt local v; v = e}''.

``{\tt P:method(E)}'' is encoded as ``{\tt P.method(P, E)}'', and
``{\tt v:method(E)}'' as ``{\tt v.method(v, E)}''.  There is still an
issue for non-primitive objects, especially when their evaluation
causes side effects, but it's always easy to write programs so that
such expressions are first stored in local variables, so that the
second encoding works.

\end{document}

