\documentclass{article}
\usepackage{amssymb}
%\usepackage{amsmath}
\usepackage[all]{xy}

\title{A Typed Lua Calculus}
\author{Fabien Fleutot}


\newcommand{\Function}[2]{\texttt{function(}#1\texttt{)\ }#2\texttt{\  end}}
\newcommand{\Return}[1]{\texttt{return\ }#1}
\newcommand{\Local}[1]{\texttt{local\ }#1}
\newcommand{\Nil}{\texttt{nil}}
\newcommand{\Table}[1]{\{\ #1\ \}}

\newcommand{\eqlabel}[1]{(\textit{#1})}
\newcommand{\compseq}[2]{\left\{ #1\ \middle/\ #2 \right\}}
\newcommand{\Leq}{\stackrel{L}{=}}

\newcommand{\evalsto}{\Rightarrow}
\newcommand{\evalstoX}[1]{\stackrel{#1}{\evalsto}}
\newcommand{\evalstoE}{\evalstoX{E}}
\newcommand{\evalstoEbar}{\evalstoX{\bar E}}
\newcommand{\evalstoS}{\evalstoX{S}}
\newcommand{\evalstoSbar}{\evalstoX{\bar S}}
\newcommand{\evalstoL}{\evalstoX{L}}
\newcommand{\evalstoLbar}{\evalstoX{\bar L}}

\newcommand{\rewritesL}{\stackrel{\T L}\Leftarrow}
\newcommand{\rewritesT}{\stackrel{\T T}\Leftarrow}

\newcommand{\T}[1]{\mathbb{#1}}
\newcommand{\Tstring}{\texttt{string}}
\newcommand{\Tboolean}{\texttt{boolean}}
\newcommand{\Tnumber}{\texttt{number}}
\newcommand{\Tnil}{\texttt{nil}}

\newcommand{\Tconst}[1]{\texttt{const }#1}
\newcommand{\Tvar}[1]{\texttt{var }#1}
\newcommand{\Tcurrently}[1]{\texttt{currently }#1}
\newcommand{\Twriteonly}[1]{\texttt{writeonly }#1}
\newcommand{\Tfield}{\texttt{field}}


\newcommand{\Tnofield}{\varnothing_{\T{F}}}
\newcommand{\Tnoreturn}{\varnothing_{\T{S}}}

\newcommand{\update}[2]{\textrm{update}(#1, #2)}
\newcommand{\trim}[2]{\textrm{trim}_{#1}({#2})}

\newcommand{\todo}[1]{
  \begin{center}
    \begin{tabular}{|p{10cm}|}
      \hline
      {\bf TODO:}~~~{\it #1}\\
      \hline
    \end{tabular}
  \end{center}}

\newcommand{\J}{\texttt{just }}
\newcommand{\C}{\texttt{currently }}
\newcommand{\V}{\texttt{var }}
\newcommand{\K}{\texttt{const }}
\newcommand{\F}{\texttt{field}}

\begin{document}
\maketitle
%-*-mode:latex; eval:(whizzytex-mode);-*-
%; whizzy-master tilo.tex

This article proposes a sound type system for the Lua programming
language. It's intended to be later combined with gradual typing and
partial type inference, so that users can blend statically and
dynamically typed program fragments, as best suits their development
needs.

Lua~\cite{lua} is a dynamic, imperative programming language, similar
in expressiveness to other modern languages such as Python, Ruby or
Javascript; compared to these, Lua specifically shines by its
embeddability, frugality in terms of hardware resources, tight
integration with C; its speed performances are also
noteworthy~\cite{perf-lua}, especially when run through a JIT
compiler~\cite{perf-luajit}. Finally, there is, by design, a proper
subset of the language which is widely acknowledged as
beginner-friendly~\cite{usability-ms}. Due to those qualities, Lua is
widely used in domains such as high-performance video games, embedded
devices or highly user-customizable systems.

Static typing is not always beginner-friendly; however, a well
designed, statically typed third party API is generally easier to use,
because many usage mistakes can be caught sooner, either when
compiling or immediately by a type-checking IDE. While introducing a
mandatory static type system in Lua would ruin several of the
language's key features, supporting optional types would significantly
improve the experience of many users.

We aim at offering such a fine-grained integration of static and
dynamic program fragments, by building upon the research on gradual
typing~\cite{gradual}. As a first step, we propose a type system,
inspired by theoretical studies of records and objects
typing~\cite{remy,sigma}, which accepts a significant proportion of
idiomatic Lua programs.

A preliminary, necessary step is the definition of a formal calculus
capturing the key semantic characteristics of Lua. Then we will
propose a type system for this calculus, which allows to check a
properly annotated program against illegal operations. We'll then hint
at ways to integrate gradual typing in this system, and partial
inference to lighten the amount of necessary annotations.

%-*-mode:latex; eval:(whizzytex-mode);-*-
%; whizzy-master tilo.tex

\section{Calculus}

This section defines a calculus, intended to capture Lua's defining
features. A compromise shall be found between faithfulness to Lua,
simplicity of the semantic rules, and ability to support a type
system.

\subsection{Notations}
\begin{itemize}

\item
  $\Sigma[x\leftarrow y]$, with $\Sigma$ a mathematical function,
  denotes the function which to $x$ associates $y$, and to all other
  values $x'$ associates $\Sigma(x')$ if defined.

\item
  $E[v\leftarrow E']$, with $E$ and $E'$ terms, and $v$ a variable
  possibly occurring in $E$, denotes the term $E$ in which all
  occurrences of $v$ are replaced with $E'$.

\item
  If $E$ denotes an element of a given kind, $\bar E$ denotes a
  sequence of such elements. A sequence has zero, one or several
  elements, which are ordered and not necessarily unique.

\item
  $\varnothing$ denotes an empty sequence. When it enhances
  understanding, it can be subscripted with the type of elements the
  sequence might have contained. For instance, an empty sequence of
  expressions might be denoted either $\varnothing$ or
  $\varnothing_E$.

\item
  $(x_n)^{\forall n\in[1...m]}$ denotes the sequence of all elements
  $x_n$ for successive values of $n$ $1, 2, ..., m$. Boundaries $1$
  and $m$ are inclusive.

\item
  Sequences are concatenated with a semicolon between parentheses:
  $(\bar E_1;\bar E_2)$. They can also be concatenated with single
  elements: $(E; \bar E)$.

\end{itemize}

\subsection{Terms definition}

In a first step, we'll introduce a calculus which captures Lua's
defining features, such as tables, functions, multi-value returns,
local variables. We will not deal with classic structures
(if/then/else statements, for loops etc.) which can be added later in
a rather straightforward way.

We also won't explicitly support features which can be easily
encoded. ``{\tt ...}'' trailing function arguments, method
invocations, globals, combined local variable declaration+assignment
are intentionally left out of the calculus.

We'll also consider it mandatory for function bodies to end with a
$\Return{\Nil}$, to save a specific rule about non-returning
functions (the Lua compiler performs this transformation for the same
reason).

Finally, we distinguish function applications as expressions $f(\bar
x)_E$ and as statements $f(\bar x)_S$: the latter discard whatever
value they might have returned. This distinction is trivial to do when
encoding a program into the calculus, and simplifies the calculus'
semantic rules.

$$
\begin{array}{rcll}
E &::=& L & \eqlabel{E-Left} \\
  &|&   P & \eqlabel{E-Primitive} \\
  &|&   E(\bar E)_E & \eqlabel{E-Apply} \\
  &|&   \Function{\bar v}{\bar S}  & \eqlabel{E-Function} \\
  &|&   \Table{ ([E^k_n]=E^v_n)^{\forall n\in[1...m]} } & \eqlabel{E-Table} \\
~\\
L &::=& v & \eqlabel{E-Variable} \\
  &|&   E[E] & \eqlabel{E-Index} \\
~\\
P &::=& \langle\textrm{string}\rangle & \eqlabel{E-String} \\
  &|&   \langle\textrm{number}\rangle & \eqlabel{E-Number} \\
  &|&   \texttt{true} & \eqlabel{E-True} \\
  &|&   \texttt{false} & \eqlabel{E-False} \\
  &|&   \texttt{nil} & \eqlabel{E-Nil} \\
  &|&   t & \eqlabel{E-TableRef} \\
  &|&   f & \eqlabel{E-ClosureRef} \\
~\\
S &::=& \Local{\bar v} & \eqlabel{S-Local} \\
  &|&   \bar L = \bar E & \eqlabel{S-Assign} \\
  &|&   E(\bar E)_S & \eqlabel{S-Apply} \\
  &|&   \Return \bar E & \eqlabel{S-Return} \\
\end{array}
$$

Expression elements are sorted into categories:
\begin{itemize}
\item $S$ denotes statements;
\item $E$ encompasses all expressions;
\item $P$ denotes primary expressions: expressions which evaluate to
  themselves, without side effect; we have $P\subset E$.
\item $L$ denotes left-hand-side values, i.e. values which can legally
  appear to the left of a $\bar L=\bar E$ assignment statement; we
  have $L\subset E$ and $L\cap P=\{\ \}$.
\end{itemize}

Table and function references $t$ and $f$ aren't part of Lua: they are
what tables and functions are evaluated to, so that the notion of
identity and mutation sharing in the calculus remain faithful to Lua.

\subsection{Semantics}

We'll now describe how calculus terms are evaluated. We'll do so in
big-step (``natural'') semantics, i.e. we give a proposition of the
form ``$\textrm{assumptions} \vdash \textrm{term to evaluate} \evalsto
\textrm{modified assumptions, fully evaluated term}$''. If no such
reduction statement can be proved, then the term to evaluate is
erroneous. This differs from small-step semantics, where one defines a
single reduction step, and the evaluation of a program is defined as
the repeated application of the reduction step until no more reduction
is possible.

More formally, the proposition $\Sigma_0, X_0\evalsto\Sigma_1, X_1$
reads ``element $X_0$, when evaluated under environment $\Sigma_0$,
reduces to $X_1$ and changes the environment into $\Sigma_1$''. To
clarify rules, this operator has been separated into $\evalstoE,
\evalstoEbar, \evalstoS, \evalstoSbar, \evalstoL, \evalstoLbar$
depending on the kind of terms on which it operates.

Rules will be presented the usual way in logic: as fraction bars, with
the premises over the bar, and the proved conclusion under
it. $a\ b\ c \over d$ means that proposition $d$ is proved if we can
provide a proof of propositions $a$, $b$ and $c$ (the ``premises''). A
rule with nothing above the line, such as ${}\over \ d\ $, is an
axiom: a proposition considered true with no need for any further
proof. A complete proof is therefore a tree of propositions, with the
goal proposition as the tree's root, and axioms as its leaves.

\paragraph{Identity of tables and closures}
In Lua, tables and functions have an identity, i.e. two structurally
equal tables are not equal, unless they're a shared reference to a
same local variable (or table element) holding the same value.  For
instance, in ``{\tt a=\{ \}; b=\{ \}}'', {\tt a} and {\tt b} are not
equal. Similarly, extensionally equal functions are not equal in the
Lua sense.

To reflect this in the calculus, we'll let ({\it E-Function}) and
({\it E-Table}) terms evaluate to a fresh reference $f$ or $t$ every
time they're evaluated; the association between the reference and its
definition will be kept in the environment store $\Sigma$. As a result,
equality works in the calculus as in Lua; for instance, in ``{\tt a=\{
  \}; b=a}'', {\tt a} and {\tt b} are indeed equal.

Moreover, in order to make mutable variables and first class functions
cohabit seamlessly, Lua defines up-values, i.e. local variables which
outlive their syntactical scope. To faithfully represent this in the
calculus, we'll perform an $\alpha$-renaming to fresh variable names
every time we encounter a ``$\Local{\bar v}$'' statement. This way,
all variables live forever in the calculus, although only up-values
might be referred to out of their scope. Put otherwise, the formal
calculus never performs garbage collection.

\paragraph{Program environment}
The program's environment, i.e. the assumptions under which a term is
evaluated, is represented by a triplet of functions $(\Sigma^L,
\Sigma^T, \Sigma^F)$. They keep track respectively of local variables'
content, tables' content and closures. Since in most cases the details
of the environment don't matter, the triplet is often shortened as
$\Sigma$.

Formally, the three functions have the following types:
\begin{itemize}
\item $\Sigma^L: v \mapsto P$ is a function from variable names 
  to the evaluated expressions they hold;
\item $\Sigma^T: (t\times P) \mapsto P$ is a function from
  table references and key expressions, to the value expressions held
  under this key in this table;
\item $\Sigma^F: f \mapsto (\bar v \times \bar S)$ is a function from
  function references $f$ to function definitions $\Function{\bar
    v}{\bar S}$.
\end{itemize}

The environment could also be used capture I/O operations, to describe
side effects other than variable value changes. This wouldn't
significantly change the calculus' semantic properties, though, and
hence will be left out.

\paragraph{Assignment to environment}
To properly define variable assignment, we'll need an operator
$\Leftarrow$ on environments, which denotes the update of what's bound
to variables and table fields in $\Sigma$. Indeed, an assignment
statement can update a mix of local variables and table contents, so
we need to describe the simultaneous update of $\Sigma^L$ and
$\Sigma^T$. The operator $(\Sigma^L, \Sigma^T) \Leftarrow (\bar L,
\bar P)$ is defined inductively as follows:
%
$$
\begin{array}{rcll}
%
(\Sigma^L,\Sigma^T) \Leftarrow (\varnothing_L,\bar P) &=& (\Sigma^L,\Sigma^T) &\eqlabel{A-$\varnothing_L$}\\ 
%
(\Sigma^L,\Sigma^T) \Leftarrow (\bar L,\varnothing_P)
&=&
(\Sigma^L,\Sigma^T) \Leftarrow (\bar L, \Nil)
&\eqlabel{A-$\varnothing_P$}\\
%
(\Sigma^L,\Sigma^T) \Leftarrow ((v;\bar L),(P;\bar P))
&=&
(\Sigma^L[v\leftarrow P], \Sigma^T) \Leftarrow (\bar L; \bar P)
&\eqlabel{A-Local}\\
%
(\Sigma^L,\Sigma^T) \Leftarrow (t[P_k], \bar L; P_v, \bar P)
&=&
(\Sigma^L,\Sigma^T[(t,P_k)\leftarrow P_v]) \Leftarrow (\bar L; \bar P)
&\eqlabel{A-Table}\\
%
\end{array}
$$

Intuitively, $\Leftarrow$ stores variable assignments in $\Sigma^L$
and table writings in $\Sigma^T$, thanks to the two last rules (the
two first ones are structural). It expects its third argument to be a
list of left-values, i.e. either variables or indexing of a primitive
term by another. As a notation facility, we'll allow to transparently
pass an extra $\Sigma^F$ argument to $\Leftarrow$. This lets use it
directly on complete environments $\Sigma$. Formally, the operator is
overloaded as follows:
%
$$(\Sigma^L,\Sigma^T,\Sigma^F) \Leftarrow (\bar L,\bar P) =
(\Sigma^L_\star,\Sigma^T_\star,\Sigma^F)
\textrm{ iff }
(\Sigma^L,\Sigma^T) \Leftarrow (\bar L, \bar P) =
(\Sigma^L_\star,\Sigma^T_\star)
$$.
%
\todo{ forbid $E_L[\Nil]=E_R$ }

\paragraph{Statements sequences evaluation}

$\varnothing_S$ denotes an empty sequence of statements. It is also the
result of a sequence which didn't return anything:

$$\frac{
}{
\Sigma, \varnothing_S \evalstoSbar \Sigma, \varnothing_S
}
\quad(\textit{ES-}\varnothing)$$

\noindent
If the first element of a sequence doesn't evaluate into a {\tt
  return}, then the result of the sequence is that of the following
statements (plus any side effect caused on $\Sigma$ by the first
statement):

$$\frac{
\Sigma, S^1 \evalstoS \Sigma_1, \varnothing
\quad
\Sigma_1, \bar S \evalstoSbar \Sigma_2, S^\star
}{
\Sigma, (S^1;\bar S) \evalstoSbar \Sigma_2, S^\star
}
\quad(\textit{ES-$\bar\varnothing$})$$

\noindent
However, if a statement evaluates to {\tt return},
the rest of the sequence isn't evaluated:

$$\frac{
\Sigma, S^1 \evalstoS \Sigma_1, \Return{\bar P}
}{
\Sigma, (S^1;\bar S) \evalstoSbar \Sigma_1, \Return{\bar P}
}
\quad(\textit{ES-}\overline{\tt return})$$

\paragraph{Statements}

\subparagraph{Return statements}
Return statements evaluate their returning values. When a function
body evaluates to $\Return{\bar P}$, it will be ``unwrapped'' back
into a $\bar P$ by ({\it EE-Apply}).
%
$$\frac{
\Sigma, \bar E \evalstoEbar \Sigma_1, \bar P
}{
\Sigma, \Return{\bar E} \evalstoS \Sigma_1, \Return{\bar P}
}
\quad(\textit{ES-Return})$$

\subparagraph{Local variables creation}
Local variable creations are immediately $\alpha$-renamed:
%
$$\frac{
\left(\Sigma^L\left[\bar w \leftarrow\overline\Nil\right],\Sigma^T,\Sigma^F\right), 
\bar S[\bar v \leftarrow \bar w]
\evalstoSbar \Sigma_1, S^\star
\quad
\bar w \textrm{\ free in }\Sigma^L
}{
(\Sigma^L,\Sigma^T,\Sigma^F), (\Local{\bar v}; \bar S)
\evalstoSbar \Sigma_1, S^\star
}\quad\eqlabel{ES-Local}$$

\noindent
The point is to handle upvalues (references to variables defined
outside of the function body) correctly. Consider for instance the
following program, featuring an up-value {\tt u}:
\begin{verbatim}
local u=1
local f = function(x)
  u=u+1
  return x+u
end
_ENV["a"], _ENV["b"] = f(1), f(2)
\end{verbatim}

\noindent
The first line {\tt local u} will be evaluated only once, and
therefore occurrences of {\tt u} in {\tt f} will all be
$\alpha$-renamed to the same fresh variable: they will indeed be
shared as expected. Conversely, in the following program:

\begin{verbatim}
local f = function(x)
  local u; u=1
  u=u+1
  return x+u
end
_ENV["a"], _ENV["b"] = f(1), f(2)
\end{verbatim}

The {\tt local u} statement will be evaluated twice, each time being
renamed in a different fresh variable, and no sharing can occur.

Evaluation of assignment contains a subtlety: the value-receiving
fields and variables, on the left of the ``$=$'' sign, must not be
fully evaluated. Instead, they need to be reduced to a ``left-normal''
form (variables or index to a table), which is done by a distinct
evaluation operator $\evalstoLbar$; once both left and right sides of
``$=$'' are evaluated, modifying the environment adequately is left to
the $\Leftarrow$ operator defined above:

$$
\frac{
\Sigma, \bar L \evalstoLbar \Sigma_1, \bar L^\star
\qquad
\Sigma_1, \bar E \evalstoEbar \Sigma_2, \bar P
}{
\Sigma_1, \bar L = \bar E
\evalstoS
(\Sigma_2\Leftarrow (\bar L^\star, \bar P)), \varnothing_S
}\quad(\textit{ES-Assign})
$$

We'll define $\evalstoLbar$ in terms of $\evalstoL$, which operates on
a single expression. Variables are considered fully evaluated when the
occur on the left of ``$=$'':

$$
\frac{
}{
\Sigma, v \evalstoL \Sigma, v
}\eqlabel{EL-v}$$

Indexed values have the table and its key evaluated, but the
field-content-accessing operation isn't performed:

$$
\frac{
\Sigma, E_T \evalstoE \Sigma_1, (P_T^1;\bar P_T)
\qquad
\Sigma_1, E_k \evalstoE \Sigma_2, (P_k^1;\bar P_k)
}{
\Sigma, E_T[E_k] \evalstoL \Sigma_2, P_T^1[P_k^1]
}\eqlabel{EL-Index}$$

With this we can easily define $\evalstoLbar$, which chains $\evalstoL$
operations by stringing their environment modifications together:

$$
\frac{
\Sigma, L_1 \evalstoL \Sigma_1, L_1^\star
\qquad
\Sigma_1, \bar L \evalstoLbar \Sigma_2, \bar L^\star
}{
\Sigma, (L_1;\bar L) \evalstoLbar \Sigma_2, (L_1^\star; \bar L^\star)
}\eqlabel{EL-$\bar L$}
\qquad
\frac{}{\Sigma, \varnothing_L \evalstoLbar \Sigma, \varnothing_L}
\eqlabel{EL-$\varnothing$}
$$

\subparagraph{Function applications in statment contexts}
Function application in a statement context is pretty similar to
function application in an expression context, except that any
returned result is thrown out.  The actual
$\beta$-reduction is therefore delegated to $\eqlabel{EE-Apply}$,
defined later:

$$ \frac{
  \Sigma, E_f(\bar E)_E \evalstoE \Sigma_1, \bar P\\
  }
  {\Sigma, E_f(\bar E)_S \evalstoS \Sigma_1, \varnothing_S}
  \quad
  (\textit{ES-Apply})$$

\paragraph{Expression sequences}
In Lua, when a single expression evaluates into several results, only
the first result is kept, except for the last one which is entirely
appended to the resulting multi-value.  For instance, if we consider
{\tt f = function(a,b,c) return a,b,c end}, the sequence {\tt
  f(10,11,12), f(20,21,22), f(30,31,32)} will evaluate to {\tt 10, 20,
  30, 31, 32}.

The rule below chains expression evaluations by stringing their
environment modifications together, and discards extraneous returned values:

$$\frac{
  (\forall n \in [1...m])\ 
  \Sigma_{n-1}, E_n \evalstoE \Sigma_n, (P_n^1;\bar P_n)
} {
  \Sigma_0, (E_n)^{\forall n\in [1...m]} \evalstoEbar \Sigma_m,
  ((P_n^1)^{\forall n\in[1...m]}; \bar P_m)
}
\quad\eqlabel{EE-Sequence}$$

\paragraph{Expressions}

\subparagraph{Variables and primitives}
Variables are replaced by their content from the store; primitives are
their own evaluation:

$$ \frac{
\Sigma^L(v) = P
}{
(\Sigma^L,\Sigma^T,\Sigma^F), v \evalstoE P
}\quad\eqlabel{EE-v}
\qquad
\frac{}{
\Sigma, P \evalstoE \Sigma, P
}\quad\eqlabel{EE-Primitive}$$

\subparagraph{Function application}
For function applications, function definitions are retrieved from
$\Sigma^F$. We define the evaluation by transforming the function
parameters into local variables, assigned to the arguments' values.
Notice that the arguments are evaluated before their assignment,
although \eqlabel{EE-Assign} would have evaluated them anyway. The
reason is, the arguments need to be evaluated outside of the
function's scope; otherwise, capture problems could occur (think for
instance of {\tt f=function(x) return x end; local x; f(x)}):

$$ \frac{
  \begin{array}{c}
    \Sigma, E_f \evalstoE (\Sigma_1^L,\Sigma_1^T,\Sigma_1^F), f\\
    \Sigma^F_1(f) = \Function{\bar v}{\bar S}\\
    (\Sigma_1^L,\Sigma_1^T,\Sigma_1^F), \bar E \evalstoEbar
    \Sigma_2, \bar P\\
    \Sigma_2, (\Local{\bar v}; \bar v = \bar P; \bar S)
    \evalstoSbar
    \Sigma_3, \Return{\bar P}
  \end{array}
  }
  {\Sigma, E_f(\bar E)_E \evalstoE \Sigma_3, \bar P}
  \quad
  (\textit{EE-Apply})$$

\subparagraph{Function creations}
As seen in \eqlabel{EE-Apply}, function definitions are retrieved from
$\Sigma^F$. They're stored in it, under a fresh name $f$, when the
{\tt function ... end} expression is found:

$$\frac{
  E_f = \Function{\bar v}{\bar S}
  \qquad
  f \textrm{ free in } \Sigma^F
}{
  (\Sigma^L,\Sigma^T,\Sigma^F), E_f
  \evalstoE 
  (\Sigma^L,\Sigma^T,\Sigma^F[f\leftarrow E_f]), f
}\quad\eqlabel{EE-Function}$$


\subparagraph{Literal tables}
To evaluate a literal table, we evaluate every key and value in order,
chaining environment modifications, then store the (key, value)
evaluated pairs in $\Sigma^T$:

$$\frac{
  \begin{array}{c}
    t \textrm{ free in } \Sigma_{2m}^T\\
    (\forall n\in[1...m])\ 
    \left\{
    \begin{array}{l}
    \Sigma_{2n-2}, E_k^n \evalsto \Sigma_{2n-1}, (P_k^n; \bar P_k^n)\\
    \Sigma_{2n-1}, E_v^n \evalsto \Sigma_{2n\phantom{-1}}, (P_v^n;
    \bar P_v^n)\\
    \end{array}\right.\\
    \Sigma^T_\star = \Sigma_{2m}^T
    [(t, P_k^n)\leftarrow P_v^n]^
    {\forall n\in[1...m]}
  \end{array}
}{
\Sigma_0, ([E_k^n]=E_v^n)^{\forall n\in[1...m]}\\
\evalstoE
(\Sigma_{2m}^L,\Sigma^T_\star,\Sigma_{2m}^F), t\\
}\quad\eqlabel{EE-Table}$$

\subparagraph{Accessing table contents}
When a value is indexed, it must evaluate to a table reference; then
the value associated with the corresponding key is retrieved from the
store $\Sigma^T$:

$$\frac{
\Sigma,E_T \evalsto \Sigma_1, t
\quad
\Sigma_1, E_k \evalsto \Sigma_2, (P_k^1;\bar P_k)
\quad
\Sigma_2^T(t, P_k^1) = P_v
} {
\Sigma, E_T[E_k] \evalsto \Sigma_2, P_v
}\quad\eqlabel{EE-Index}$$

In Lua, a key which has never been set in a table is associated with
value \Nil. To reflect this, an evaluation must start with all
table/key pairs associated with $\Nil$:

$$(\forall t)(\forall P)\ \Sigma^T(t, P) = \Nil$$

\subsection{How evaluation can fail}

The operational semantics above defines the result of programs, as
long as they:
\begin{itemize}
\item don't get stuck in infinite recursion;
\item don't try to index a non-table (cf. premises of
  $\eqlabel{EE-Index}$, which is the only evaluation rule applying to
  terms of the form $E[E]$, and requires the indexed object to be a
  reference to a table);
\item don't try to apply a non-function (cf. premises of
  $\eqlabel{EE-Apply}$, which is the only evaluation rule applying to
  terms of the form $E(\bar E)$, and requires the applied object to be
  a reference to a function;);
\item always return a value from a function, i.e. all function bodies,
  when parameters are substituted with arguments, evaluate to a
  $\Return{\bar P}$ statement value. It's easily proved that by
  appending a $\Return{\Nil}$ at the end of the function's body, a
  function body evaluating to $\varnothing_S$ will evaluate to
  $\Return{\Nil}$ instead.
\end{itemize}

The last point is very easily addressed, and the first one is well
known as undecidable; the most reasonable definition of static
correctness, for a program, is to provably perform no indexing of a
non-table value, and no function call on a non-function. A sound type
system for the present calculus will provide formal proofs that a
given term cannot involve such incorrect sub-terms. The design of such
a type system is the subject of the next section.

%-*-mode:latex; eval:(whizzytex-mode);-*-
%; whizzy-master tilo.tex

\section{Static type system}

In the previous section, we've seen that we wanted a type system to
prevent indexing of non-tables as well as application of
non-functions. We've also mentioned that a type system which
eliminates all incorrect terms will also eliminate some correct ones
(a direct consequence of the calculus' T\"uring-completeness, easily
demonstrated by encoding the $\lambda$-calculus in it). This section
will try to find a reasonable compromise: a type system which accepts
a lot of ``reasonable'' terms, catches all incorrect ones, and doesn't
force too much book-keeping on users.

\subsection{Notations}
%\begin{itemize} \item 
$<:$ denotes subtyping. $\T T_1 <: \T T_2$ means that
$\T T_1$ is a subtype of $\T T_2$, i.e. a term of type $\T T_1$ can
be used everywhere a term of type $\T T_2$ is expected.
%\end{itemize}

\subsection{Type System}

$$\begin{array}{rcll}
\T E &::=& [\overline{P:\T F}| \T F]  &\eqlabel{TE-Table}\\
&|& \T P & \eqlabel{TE-Primitive}\\
&|& \bar\T E \rightarrow \bar\T E &\eqlabel{TE-Function}\\
%&|& v &\eqlabel{TE-Variable} \\
&|& \top &\eqlabel{TE-Top} \\
\\
\T P &::=& \Tnil\ |\ \Tboolean\ |\ \Tnumber\ |\ \Tstring\\
\\
\T F &::=&  \J{\T E} &\eqlabel{TF-Just} \\
&|&  \C{\T E} &\eqlabel{TF-Currently} \\
&|& \V{\T E} &\eqlabel{TF-Var} \\
&|& \K{\T E} &\eqlabel{TF-Const} \\
&|& \F &\eqlabel{TF-Field} \\
\\
\T S &::=& \Return{\bar\T E} &\eqlabel{TS-Return}\\
&|& \Tnoreturn &\eqlabel{TS-None}\\
\end{array}$$


\paragraph{Lua key features to respect}

The type system should capture as many Lua-specific idioms as
possible. Tables are of course central in Lua, and quite similar in
some respects to objects of calculi such as Abadi \&
Cardelli's~\cite{sigma}, Fisher Honsell \& Mitchell~\cite{fhm}, or
R\'emy's~\cite{remy}. Among others, they are defined with no native
notion of classes. The most striking Lua-specific features are:

\begin{itemize}
\item Lua tables---and therefore Lua itself---are deeply
  imperative. There are primitives in Lua to alter a table, but
  neither to copy nor to functionally update it; despite closures and
  tail-call optimization, idiomatic Lua code is not functional. This
  contrasts with most theoretical studies of object-oriented calculi,
  which inherit from $\lambda$-calculus a preference for functional
  primitives and idioms.
\item Lua tables are arbitrary value$\rightarrow$value hashtables,
  whereas object calculi typically index object fields with labels
  taken from a separate (enumerable) set. In the first version of our
  type system, we'll only type tables whose keys have primitive types
  \verb+string+, \verb+number+ or \verb+boolean+. Homogeneously typed
  hashtables should be easy to add at a later stage---they mostly
  behave as simplified functions, type-wise; but arbitrarily mixed
  tables, acting as a raw mix of records and hashtables, are untypable
  in general.
\item Lua makes a distinction between statements and expressions. This
  means that two distinct type kinds $\T S$ and $\T E$ are
  defined. This is again in contrast with most theoretical calculi,
  which are purely expression-based.
\item There's no notion of ``undefined label'': all table keys except
  \verb+nil+ are defined in all tables; keys which haven't been
  explicitly assigned are associated with the value \verb+nil+.
\item A key consequence is that an table's type changes during its
  lifetime: when created all its fields have type \verb+nil+, then
  these field types are modified as meaningful values are put in
  them. Whereas most calculi allow to add fields to object, ours will
  allow to change their type.
\item Lua functions take multiple arguments, and return multiple
  values.
\item Expressions can be collected into expression sequences, to be
  used in assignments, function calls and function results. They get
  their dedicated type kind $\bar\T E$.
\end{itemize}

Beyond those specific needs, the type system will include many staples
of modern type systems, such as structural subtyping, functions
contravariant in their arguments and covariant in their results,
covariant read-only table fields, invariant read+write fields.

Types will not be nullable (there won't be any legal way to derive
$\Nil<:\T E$ for any $\T E$ other than \Nil): catching ``NPE'' ({\em
  Null Pointer Exceptions}, as nicknamed by traumatized Java
developers) has been done for ages in ML-inspired languages; Hoare
himself, who first introduced implicitly nullable types in Algol,
called them ``{\em my billion dollar mistake}''~\cite{?}.  However,
since nullable function arguments and results are an important idiom
in Lua, future versions of the calculus will have to support either an
explicit \verb+nullable+ type modifier, or a generic union type, allowing
to type e.g. an optional number as \verb+nil|number+.

\paragraph{Tables}
Lua tables accept all values as keys except \Nil, and there's no
notion of unset/undefined key; it's legal to request {\tt foo["bar"]}
even if the key {\tt"bar"} has never been set in {\tt foo}: it will
return \Nil. So in practice, all but a finite set of keys in a given
table will return a value other than \Nil. To reflect this in the type
system, a table type will contain a default field type, shared by all
but a finite number of its fields; this type will usually be set to
some variant of \Nil. The other key/type pairs are listed explicitly.
For instance, $[\texttt{"x"}:\K{\Tnumber}|\K\Nil]$ describes a table
such as \verb+{["x"]=1}+, with a field {\tt"x"} of type $\K\Tnumber$,
and all other fields left to \Nil.

In the current version, field keys are limited to values of atomic
types \verb+string+, \verb+number+ or \verb+boolean+. Tables using
other keys will not be typable statically. Some future extensions are
possible, and will be studied separately.

In contrast with type systems inspired by Abadi \& Cardelli's, we
won't introduce a $\zeta(s)$ self-type binder in the type
system~\cite{sigma}: it substantially complicates it, mostly to allow a
functional-style use of objects that doesn't seem to correspond to any
widespread Lua idiom.

Table types are considered equal modulo fields reordering, and
expansion of the default field type; the following types are all
considered equal:

$$\begin{array}{ll}
 \phantom{{}={}}
    [P_1: \T F_1; P_2:\T F_2|\T F_d]\\
 {}=[P_2: \T F_2; P_1:\T F_1| \T F_d] & \eqlabel{reordering}\\
 {}=[P_1: \T F_1; P_2:\T F_2; P_3:\T F_d| \T F_d]
    & \eqlabel{default expansion} \\
\end{array}$$

Moreover, it's illegal for a field key to appear more than once in the
same table: $[P_1:\T F_1; P_1:\T F_2 | \T F_d]$ is not a well-formed
type.

Finally, we'll admit as a shortcut that $[\overline{P:\T F}]$ means
$[\overline{P:\T F}|\F]$.

\paragraph{Field types}

Field types are expression types with a prefix modifier: $\J\T E$,
$\C\T E$, $\V\T E$, $\K\T E$, and simply $\F$ without a type
parameter. They give control over field variance, i.e. they prevent
some operations, but in exchange allow more permissive subtyping, and
hence allow to use tables in more contexts. \verb+var+, \verb+const+
and \verb+field+ will be familiar to people who studied structural
subtyping, and offer the expected variance properties. $\J\T E$ and
$\C\T E$ are more unusual, and allow to change a value's type for an
unrelated one, under specific conditions.


\subparagraph{Read-write fields}
$\V\T E$ is the type of a field which can be read and written with
values of type $\T E$. It's not covariant, i.e. even if $\T E_1<:\T
E_2$, we don't have $[P:\V\T E_1]<: [P:\V\T E_2]$. To see why, let's
consider the subtyping relationship $\texttt {positive} <: \texttt
{number}$ \footnote{Positive numbers are numbers, but not the other
  way around. This example of subtyping has been chosen because it's
  hopefully familiar and intuitive for everyone; but it's only
  intended to illustrate variance issues, and the calculus won't have
  a specific {\tt positive} type.}. If we had $[P:\V\texttt
    {positive}] <: [P:\V\texttt {number}]$, we could take a
table of type $[P:\V\texttt {positive}]$, partially
forget its type through subtyping into $[P:\V\texttt
    {number}]$, then write a negative number in its field
$P$. Other parts of the program, which retained the more precise
type $[ P: \V\texttt{positive}]$ for the same
table, might break because they take for granted that $P$'s
content is positive.

\subparagraph{Read-only fields}
If we promise not to overwrite a field, however, we can make it
covariant. This is the purpose of $\K\T E$ fields: the type system
will prevent from updating such fields, but in exchange, whenever we
have $\T E_1<:\T E_2$, we also get $\K\T E_1<: \K\T E_2$. Adapting the
previous example, $[P: \K\texttt {positive}]$ is a subtype of $[P:
  \K\texttt {number}]$, because when reading its $P$ field, one
gets a {\tt positive} which is indeed a {\tt number}; since we can't
write in it, there's no danger of putting a negative number where the
type system expects to find only positive ones.

\subparagraph{Contravariant fields}
Symmetrically, we could promise not to read a field, and get
contravariance in exchange (whenever $\T E_1<:\T E_2$, we get
$[P:\texttt{writeonly }\T E_2]<:[P:\texttt{writeonly }\T E_1]$). However,
this seems of limited practical use, so we'll leave this out of the
type system.

\subparagraph{All fields} If we promise neither to read nor
to write a given field, it becomes bivariant: whether $\T E_1<:\T E_2$
or $\T E_2<:\T E_1$, or even if $\T E_1$ and $\T E_2$ are
incomparable, we'd still have $\F\ \T E_1 <: \F\ \T
E_2$. We'll therefore simply write it $\F$: no need to keep $\T
E$ in it, since it isn't used anyway. It's the super-type of all other
type fields, and it acts as the {\tt private} modifier does in C++
inspired languages: you can neither override nor use a field of this
type.

\subparagraph{Type-changing field types}

$\C\T E$ means that a field currently has type $\T E$, but that this
type can be changed without breaking the program. This is an
unusually liberal typing rule, and as such, it will only be allowed
under strictly controlled circumstances. Most notably, a table type
with some \verb+currently+ fields will have to be used linearly: if
several variables allowed access to the same \verb+currently+ field, one
variable could change the field's content type without the other
variable's knowledge, and break the program in unpredictable
ways. Hence, it will be mandatory to weaken the type of $\C\T
E$ fields into \F, before using them in non-linear ways, thus
preventing both read and write operations on it.

$\C\T E$ field types are intended to allow idioms such as ``{\tt
  x=\verb+{ }+;x.f\_1= E1; ...; x.f\_n=En}''; the type of \verb+x+ in
this program will change at each statement of this sequence, from
$[|\C\Nil]$ to $[\texttt{"f\_1"}:\C\T E_1; ...; \allowbreak
  \texttt{"f\_n"}:\C\T E_n|\C\Nil]$.

We mentioned a criterion of linearity: there must be at most one
reference to a $\C\T E$ field. Otherwise, one reference might change
the type of the field's content without the other references'
knowledge. The typing rules will be designed in such a way that
whenever extra references to it are created, these will be typed as
\F, i.e. inaccessible.

\subparagraph{Unreferenced field types}

Finally, we need a type indicating that an object is completely
unreferrenced, and can therefore be stored safely into a $\C\T E$
field. For instance, in {\tt x=\{foo=1\}}, if the right-hand-side
table was typed $[\texttt{"foo"}:\C\Tnumber| \allowbreak \C\Nil]$, it
would have to be weakened into $[|\F]$ before being stored in $x$, in
case there was already a reference to it. 

Therefore, we distinguisg $\C\T E$ the type of a field referenced
once, and $\J\T E$ the type of a field which isn't referenced at
all. In the example above, the right-hand-side is typed
$[\texttt{"foo"}:\J\Tnumber|\J\Nil]$, and weakened into
$[\texttt{"foo"}:\C\Tnumber| \allowbreak \texttt{cur}\-\texttt{rently
  }\Nil]$ when stored into \verb+x+. A further \verb+y=x+ statement
would see the type stored in \verb+y+ weakened to $[|\F]$.

\subsection{Subtyping rules}
The subtyping relationship is a partial order, defined as the smallest
transitive closure of the rules listed in this subsection.

\paragraph{Structural rules}
The subtyping relationship, defined over expression, field and
statement types, is reflexive and transitive; $\top$ is the biggest
expression type:

$$
\frac{}{
\T E <: \top
}\eqlabel{$<:\top$}
$$
$$
\frac{}{\T E<:\T E}
\qquad
\frac{}{\T F<:\T F}
\qquad
\frac{}{\T S<:\T S}
\qquad
\eqlabel{$<:$Refl}
$$
$$
\frac{\T E_1<:\T E_2 \quad \T E_2 <: \T E_3}{\T E_1<:\T E_3}
\quad
\frac{\T F_1<:\T F_2 \quad \T F_2 <: \T F_3}{\T F_1<:\T F_3}
\quad
\frac{\T S_1<:\T S_2 \quad \T S_2 <: \T S_3}{\T S_1<:\T S_3}
$$
$$
\eqlabel{$<:Trans$}
$$


\paragraph{Fields subtyping}

We have $\V\T E<:\K\T E$, \verb+const+'s covariance, and
\F\ the top field type:

$$
\frac{}{
{\T F} <: \F
}\eqlabel{$<:$Field}
$$
$$
\frac{}{
\V{\T E} <: \K{\T E}
}\eqlabel{$<:$Const}
%
\qquad
%
\frac{
\T E_1 <: \T E_2
}{
\K{\T E_1} <: \K{\T E_2}
}\eqlabel{$<:$Const${}^+$}
$$
$$
\frac{}{
\J\T E<:\C\T E}
\eqlabel{$<:$Currently}
\qquad
\frac{}{
\J\T E<:\V\T E}
\eqlabel{$<:$Var}
$$
$$
\frac{
\T E_1 <: \T E_2
}{
\J{\T E_1} <: \J{\T E_2}
}\eqlabel{$<:$Just${}^+$}
$$

We do {\em not} have $\C\T E<:\V\T E$. Indeed, mutable field types are
not a special case of variable fields: the latter can be used with
less restrictions when linearity cannot be guaranteed. For instance,
if $\texttt{x}: [\V\Tnumber]$, its content can be assigned to \verb+y+
with ``\verb+x=y+'', and \verb+y+ will also have type $[\V\Tnumber]$.
However, if \verb+x+ had type $[\C\Tnumber]$, \verb+y+ would only get
type $[\F]$, because the following statement might be e.g. ``{\tt
  x=false}'': one cannot count on \verb+y+ keeping its \Tnumber\ type.

\paragraph{Functions subtyping}
Functions are contravariant in their arguments, and covariant in their
results:
$$
\frac{
\bar\T E_?^2 <: \bar\T E_?^1
\quad
\bar\T E_!^1 <: \bar\T E_!^2
}{
\bar\T E_?^1 \rightarrow \bar\T E_!^1
<:
\bar\T E_?^2 \rightarrow \bar\T E_!^2
}\eqlabel{$<:$Function}
$$


\paragraph{Tables subtyping}

Subtyping between tables is directly lifted from field subtyping;
unreferenced tables, marked with a prime, can be considered as
regular table whenever suitable.

$$\frac{
  (\forall n\in[0...m])\
  \T F_n^a <: \T F_n^b
}{
  [(P_n:\T F_n^a)^{\forall n\in[1...m]}|\T F_0^a] <:
  [(P_n:\T F_n^b)^{\forall n\in[1...m]}|\T F_0^a]
}\eqlabel{$<:$Table}
$$

% probably no need for structurel subtyping inside linear tables.

\paragraph{Expression sequences subtyping}
Subtyping between expression sequences is only defined between
sequences of the same length. Typing rules will pad sequence with
\verb+nil+s on the right whenever appropriate:

$$\frac{ (\forall n\in[1...m])\ \T E_1^n <: \T E_2^n }{
  (\T E_1^n)^{\forall n\in[1...m]} <: (\T E_2^n)^{\forall
    n\in[1...m]} }\eqlabel{$<:\bar\T E$}$$

\paragraph{Statements subtyping}
Statement types are either $\Tnoreturn$, or of the form $\Return\bar\T
E$. In the latter case, subtyping is lifted from
expression sequences subtyping:

$$
\frac{
\bar\T E_1 <: \bar\T E_2
}{
\Return{\bar\T E_1} <:  \Return{\bar\T E_2}
}\eqlabel{$<:$Return}
$$


%-*-mode:latex; eval:(whizzytex-mode);-*-
%; whizzy-master tilo.tex

\subsection{Typing rules}

This section gives the typing rules, which allow to determine the
belonging of terms to certain types. To do that, we'll enrich the
calculus with a couple of type annotations. We won't discuss the
possibility to algorithmically infer some of those.

\paragraph{Typing environments}

The rules exposed below use typing environments $\Gamma: v\mapsto\T
F$, functions from variables to field types (rather than, as one could
have expected, expression types). Indeed, being able to separate
constants from variables in the type system is valuable, and more
importantly, the linearity issues handled by the \verb+currently+
modifier occur with local variables as well as with table fields.

This section won't address variable shadowing
issues\footnote{i.e. homonymies, as in ``{\tt local x; x=function(x)
    return x end}'', where there are two distinct variables which both
  share the name {\tt x}.}: since we work on static terms which we
don't evaluate, an appropriate $\alpha$-renaming before typing can
ensure that no such shadowing occurs.

\paragraph{Notations}
\begin{itemize}
\item $\Gamma[v \leftarrow \T F]$ is the function which, to $v$,
  associates $\T F$, and to all other values $w\in\textrm{dom} (\Gamma)
  \backslash \{v\}$ associates $\Gamma(w)$.
\item This definition is extended homomorphically to sequences of
  variables and values $\Gamma [\bar v \leftarrow \bar\T F]$.
\item The empty environment, i.e. the function with an empty domain,
  is written $\varnothing_\Gamma$.
%% \item $\Gamma_1\cup\Gamma_2$ is the function defined over
%%   $\textrm{dom}(\Gamma_1) \cup \textrm{dom}(\Gamma_2)$ which, to $v$,
%%   associates $\Gamma_1(v)$ or $\Gamma_2(v)$, when
%%   $\textrm{dom}(\Gamma_1) \cap \textrm{dom}(\Gamma_2) =
%%   \varnothing_\Gamma$.
%%
%% \item $\T F_1\lhd\T E=\T F_2$ means that it's legal to store an
%%   expression of type $\T E$ in a field of type $\T F_1$, and that
%%   afterward, the field's type becomes $\T F_2$. If the final type
%%   doesn't matter, the notation can be abridged into $\T F_1\lhd\T E$.
%% \item $(\T E_1, P)\lhd\T E = \T E_2$ is a variant of the above,
%%   operating on expression types. It means that an object of type $\T
%%   E_1$ can support an update of its field $P$ with an expression of
%%   type $\T E$, and that the receiving object's type becomes $\T E_2$.
\item $\Gamma\vdash T:\T T$ means that under the assumptions in
  environment $\Gamma$, term $T$ has type $\T T$.
\item The notation $\Gamma\vdash E\therefore\T F$ describes the types
  of left-hand sides operands in ``$=$'' assignment statements; these
  variables and table fields must retain field types rather than
  expression types, because their field qualifier indicates whether
  and how they can be updated.  It means ``under the assumptions
  $\Gamma$, and in an assignment's left-hand side context, expression
  $E$ has type $\T F$''.
\item We'll refer to expressions which can syntactically appear on the
  left-hand-side of an assignments, and in a ``$\therefore$''
  judgment, as ``slots''. Those are local variables
  \eqlabel{E-Variable} and indexed tables \eqlabel{E-Index}.
%% \item $\T Q$ denotes operators which prefix expression types $\T E$ 
%%   with annotations \verb+currently+, \verb+var+, \verb+const+, or
%%   simply return field type \verb+field+.
\end{itemize}

\paragraph{Calculus extensions}
The untyped calculus is extended with a couple of annotations which
will allow to insert typing hints at appropriate places in programs.
Syntactically, typing annotations will make heavy use of the pound
$\#$ ascii character. This character, unused in Lua where type
annotations may occur, will hopefully make typing annotations stand
out visually, and make it easy to preprocess them out of a program's
sources, so that it can be used by other interpreters.

\begin{itemize}
\item statement $\#\Return{\bar{\T E}}; \bar S$ weakens the type of
  statements sequence $\bar S$ to statement type $\Return\bar\T E$.
\item $\Function{\overline{v\ \# \T E}}{\bar S}$ allows to give type
  annotations to function parameters, which are notoriously hard to
  infer effectively.
%% \item $\#v:\T F;\bar S$ allows to weaken the type of variable $v$ to
%%   $\bar F$ when typing the remaining of the statements sequence $\bar
%%   S$. This is especially useful to weaken \verb+currently+ fields into
%%   \verb+const+ or \verb+var+ ones, so that they can be used in
%%   non-linear conditions.
\item Left-hand sides of assignments take a type field, allowing to
  handle \verb+currently+ field type updates, and linearity
  issues: $\overline{L\ \#\T F} = \bar\T E$.
\end{itemize}
%
$$
\frac{}{\Sigma,(\#\Return{\bar\T E};\bar S) \Rightarrow \Sigma, \varnothing_S}
%
\quad
%
\frac{\Sigma_1, \Function{\bar v}{\bar S} \Rightarrow \Sigma_2, f}
{\Sigma_1, \Function{\overline{v\ \#\T E}}{\bar S} \Rightarrow \Sigma_2, f}
$$
$$
\frac{
\Sigma_1, (\Local \bar L=\bar E;\bar S) \evalsto \Sigma_2, S
}{
\Sigma_1, (\Local \overline{L\ \#\T F}=\bar E;\bar S) \evalsto \Sigma_2, S
}
$$

%%%% Wrong: it only allows to change #currently types.
%% Notice that a dummy assignment $L_1\ \#\T F_1=L_1$, while not
%% having any side effect, allows to retype the variable/field $L_1$,
%% provided that the environment $\Gamma$ allows it. In an actual
%% implementation of tha calculus, some syntax sugar for this use case
%% would most likely be desirable.

\paragraph{Expression sequences}
Expressions in Lua can evaluate into multiple values. When
concatenating expressions in a sequence, Lua only keeps the first
value of each expression's evaluation, except for the last one which
is expanded \eqlabel{EE-Sequence}.  For instance, if we have
$\texttt{a}:(\T E_a^1;\T E_a^2)$, $\texttt{b}:(\T E_b^1;\T E_b^2)$
and $\texttt{c}:(\T E_c^1;\T E_c^2)$, then the type of the sequence
$(\texttt{a}; \texttt{b}; \texttt{c})$ is $(\T E_a^1; \T E_b^1;
\T E_c^1; \T E_c^2)$. A noteworthy property is that the number of
elements in the type might be bigger than the number of expressions in
the sequence. Another is that appending \Nil types at the end of a
sequence type doesn't change it: {\tt number, number, nil} must be
treated as equal to {\tt number, number}. This will be ensured by
every rule effectively combining expression type sequences.

$$
\frac{
(\forall n\in[1...m]) \quad \Gamma\vdash E_n:(\T E_n^1;\bar\T E_n)
}{
\Gamma\vdash \bar E:(\T E_n^1)^{\forall n\in[1...m]};\bar\T E_m
}\eqlabel{TR-$\bar E$}
$$

\paragraph{Primitive expressions}

$$
\frac{}{\Gamma\vdash \langle\textrm{number}\rangle:\Tnumber}
\quad
\frac{}{\Gamma\vdash \langle\textrm{string}\rangle:\Tstring}
$$
$$
\frac{}{\Gamma\vdash \texttt{true}: \texttt{boolean}}
\quad
\frac{}{\Gamma\vdash \texttt{false}: \texttt{boolean}}
$$
$$\eqlabel{TR-P}$$

%% \todo{We might have to type t and f table/function references too, in
%%   order to get some subject reduction lemma. Not sure how to do that
%%   in big-step semantics, though, as I won't get any Felleisen-style
%%   syntactic criterion.}

\paragraph{Statement sequences} 
Statement sequences appear in function bodies. What we need to know
about them, besides the fact that they don't fail during evaluation,
is the type of the expression sequences they return. Therefore we have
two families of statement types: $\Tnoreturn$ for terms which don't
return, and $\Return{\bar\T E}$ for terms returning a sequence of
expressions of type $\bar\T E$.

$$
\frac{
\Gamma\vdash S: \Tnoreturn
\quad
\Gamma\vdash \bar S: \T S
}{
\Gamma\vdash (S;\bar S): \T S
}\eqlabel{TR-$\Tnoreturn$}
%
\qquad
%
\frac{
\Gamma\vdash S: \Return{\bar\T E}
}{
\Gamma\vdash (S;\bar S): \Return{\bar\T E}
}\eqlabel{TR-{\tt return}}
$$
%
$$
\frac{
\Gamma\vdash\bar S:\Return{\bar\T E_1}
\quad
\bar\T E_1 <: \bar\T E_2
}{
\Gamma\vdash(\#\Return{\bar\T E_2}; \bar S):\Return{\bar\T E_2}
}
\eqlabel{TR-\#{\tt return}}
$$
\paragraph{Variables}
Variable types are remembered in the environment $\Gamma$. They're
slots, and have a field type $\T F$ rather than an expression type $\T
E$. This allows to use them on the left-hand-side of assignements, to
remember whether they're constant, private, or whether their current
type can be updated. Field type judgments use the operator
``$\therefore$'', to avoid being confused with expression type
judgments using ``:''.

$$
\frac{
\Gamma(v) = \T F
}{
\Gamma \vdash v \therefore \T F
}
\eqlabel{TR-$\therefore$}
$$

When a slot $L$ is used as a normal expression rather than an
assignment's left-hand-side, its field type can be projected into an
expression type:

$$
\frac{
\Gamma \vdash L \therefore \C\T E
}{
\Gamma \vdash L:\T E
}
\quad
\frac{
\Gamma \vdash L \therefore \V\T E
}{
\Gamma \vdash L:\T E
}
\quad
\frac{
\Gamma \vdash L \therefore \K\T E
}{
\Gamma \vdash L:\T E
}
\quad\eqlabel{TR-L}
$$

There's no rule to project type \F: it's never legal to use such a
field in an expression's context. We'll see that no rule to project
\verb+just+ types is needed either, because there's no legal way to
derive a \verb+just+ type for a left-hand=side expression.

\paragraph{Table fields} 
In this rule, the variable $\phi$ denotes a set of key/field types,
plus the table's default type.

$$\frac{
\Gamma\vdash E_T: [P: {\T F_K};\phi]
}{
\Gamma\vdash E_T[P] \therefore \T F_K
}\eqlabel{TR-$[\ ]$}$$

In some cases, an expansion of the default field type might be needed,
e.g. we have ${\Gamma\vdash E:[|\K\Nil]} \over {\Gamma\vdash
  E[\texttt{"x"}] \therefore \K\Nil}$, because $[|\K\Nil] =
[\texttt{"x"}:\K\Nil|\K\Nil]$.

\paragraph{Literal tables}
Literal tables have all their fields typed with \verb+just+ modifiers.
When the literal table will be stored in a variables, the field types
will be weakened into either {\tt currently}, {\tt var}, {\tt const}
or {\tt field} types.

$$
\frac{
(\forall n\in[1...m]) \quad \Gamma\vdash E_n:(\T E_n^1;\bar\T E_n)
}{
  \Gamma\vdash \{([P_n] = E_n)^{\forall n\in[1...m]}\}:
  [(P_n:\J{\T E_n^1})^{\forall n\in[1...m]}|\J\Nil]
}$$
\vspace{-2em}
\begin{flushright}
  $\eqlabel{TR-Table}$
\end{flushright}
 


\paragraph{Local variable declarations}

Unlike in most type systems, newly created variables are given the
\Nil\ type, rather than the type of their future content. This is
because assignment statements change the type of the variables on
which they operate, as we'll see below. Besides, typing as non-\Nil\ a
variable while it contains \Nil\ wouldn't be sound.

$$
\frac{
\Gamma[\bar v\leftarrow\overline{\C\Nil}]\vdash S:\T S
}{
\Gamma\vdash (\Local{\bar v}; \bar S) : \T S
}\eqlabel{TR-Local}
$$

\paragraph{Assignments}

Assignments can change the type of \verb+currently+ variables and
fields in $\Gamma$; they can also perform weakenings, essentially
changing \verb+var+ field types into \verb+const+s. They must be
preventing from altering \verb+const+ and \verb+field+ slots.  To type
them, we'll need two auxiliary predicates:

\begin{itemize}
\item $\Gamma\vdash\update{L}{\T F}=[\sigma]$ checks whether
  variable/field $L$ is allowed to have its current type changed into
  $\T F$. If it is, it returns a substitution $[\sigma]$ over
  environments, so that $\Gamma[\sigma]$ is the typing environment in
  effect after the assignment has been performed.
\item $\T F_L \rhd \T F_R$ checks whether the content of slot of type
  $\T F_R$ can be stored in a slot of type $\T F_L$. It is, as we'll
  see, a subset of $:>$ the opposite of the subtyping relationship.
\end{itemize}

The former will prevent from changing the type of \verb+var+ fields,
and from changing the content of \verb+field+ or \verb+const+ fields:
there will be no rule allowing to derive $\Gamma\vdash\update{L}{\K\T
  E}=[\sigma]$; moreover, $\Gamma\vdash\update{L}{\V\T E}=[\ ]$ will
only be derivable from $\Gamma\vdash L\therefore\V\T E$, and will only
produce empty type substitutions $[\ ]$. Once update() has allowed an
assignment based on the field's former and new types, $\rhd$ checks
that the content put in the field is consistent with the new type, to
prevent such unsound assignments as {\tt v \#var number="abc"}.

The following rule allows to change the content of a \verb+var+ slot,
as long as its type isn't changed:

$$\frac
{\Gamma\vdash L \therefore \V\T E}
{\Gamma\vdash \update{L}{\V\T E} = [\ ]}
\eqlabel{UP-{\tt var}}
$$


\verb+currently+ slots can change the type of the value they contain,
but the slot type itself can also be changed, into a \verb+var+, a
\verb+const+ or even a \verb+field+.

$$\frac
{\Gamma\vdash v \therefore \C\T E}
{\Gamma\vdash \update{v}{\T F} = [v\leftarrow\T F]}
\eqlabel{UP-{\tt cur}}
$$

\verb+currently+ fields within tables pose an additional difficulty:
if the field's type changes, the type of the table containing it also
changes. Therefore, the table itself must also be stored in a
\verb+currently+ slot, etc.\ recursively until we reach a top-level
\verb+currently+ variable. The typing of assignments to those fields
is therefore defined recursively, with \eqlabel{UP-{\tt cur}} as a
base case, and \eqlabel{UP-{\tt cur}$[\ ]$} below as the inductive
rule:
%
$$\frac
{ 
  \begin{array}{c}
    \Gamma\vdash L \therefore \C[P:\C\T E; \phi]\\
    \Gamma\vdash \update{L}{\C[P:\T F; \phi]} = [\sigma]\\
  \end{array}
} {
  \Gamma\vdash\update{L[P]}{\T F} = [\sigma]
}
\eqlabel{UP-{\tt cur}$[\ ]$}
$$


As a usage example, let's consider an object \verb+x+ with a field
\verb+y+ currently containing a number, and updated to a string
variable. To make the proof tree terser, we'll use the following
definitions for \verb+x+'s former type $\T F_1$, its new type $\T
F_2$, and the typing environment before assignment $\Gamma$
respectively:
%
$$
  \begin{array}{l}
    \T F_1 = \C[\texttt{"y"}:\C\Tnumber]\\
    \T F_2 = \C[\texttt{"y"}:\V\Tstring]\\
    \Gamma = \{\texttt{x}\mapsto\T F_1\} \\
  \end{array}
$$
%
The soundness of environmnent substitution $[x\leftarrow\T F_2]$ is
computed by \eqlabel{UP-{\tt cur}} over {\tt x}; from there, it's
concluded by \eqlabel{UP-{\tt cur}$[\ ]$} that \verb+x["y"]+ can also
cause this substitution, because both \verb+x+ and \verb+x["y"]+ are
\verb+currently+ slots:

$$\frac
{
          \frac
              {\displaystyle \Gamma(\texttt{x}) = \T F_1}
              {\displaystyle \Gamma\vdash \texttt{x} \therefore \T F_1}
              \eqlabel{TR-$\therefore$}
    \qquad
    \frac
        {\displaystyle
          \frac
              {\displaystyle \Gamma(\texttt{x}) = \T F_1}
              {\displaystyle \Gamma\vdash \texttt{x} \therefore \T F_1}
              \eqlabel{TR-$\therefore$}
        }
        {\displaystyle \Gamma\vdash\update {\texttt{x}}{\T F_2} = [x\leftarrow \T F_2]}
        \eqlabel{UP-{\tt cur}}
}
{ \Gamma\vdash\update{\texttt{x["y"]}}{\T F_2} = [x\leftarrow\T F_2] }
\eqlabel{UP-{\tt cur}$[\ ]$}
$$

Because \verb+currently+ fields are only usable when they're inside
other \verb+currently+ fields all the way up to a variable, there's no
point having types such as $v \therefore \V[P:\C\T E; \phi]$: it
wouldn't allow anything more than $v \therefore \V[P:\V\T E; \phi]$.

~

$\T F_L \rhd \T F_R$ checks whether what's stored in a field has an
appropriate expression type. It also keeps track of linearity, forcing
to transform $\J\T E$ into $\C\T E$, and $\C\T E$ into $\F$. The
relation is expressed between two fields rather than a field and an
expression, to ease its recursive over tables (last rule below):

$$
\frac  {}  {\C\T E \rhd \J\T E}
%
\quad
%
\frac  {}  {\F \rhd \C\T E}
$$
$$
\frac  { }  {\V\T E \rhd \V\T E}
%
\qquad
%
\frac  { }  {\K\T E \rhd \K\T E}
%
\quad
%
$$
$$
\frac
{(\forall n\in[0...m]) \quad \T F_n^L \rhd \T F_n^R}
{
\C[(P_n:\T F_n^L)^{\forall n\in[1...m]}|\T F_0^L]
\rhd
\J[(P_n:\T F_n^R)^{\forall n\in[1...m]}|\T F_0^R]
}
$$
$$
\frac
    {\T F_L \rhd \T F_{R1} \quad \T F_{R1} :> \T F_{R2}}
    {\T F_L \rhd \T F_{R2}}
$$
$$\eqlabel{Accept}$$

Notice that although $\rhd$ is a subset of $:>$, it isn't an order
relayionship: it isn't idempotent (e.g. $\J\T E\not\rhd\J\T E$). By
using the composition with $:>$, we can choose to store a \verb+just+
field inside a table into either a \verb+currently+ or a \verb+var+
one; the former will allow to change the field's type, but any copy of
it can't be used (it will have to be further weakened into
\verb+field+); the latter will lock the type's content, but allows to
make and use further copies.

It is possible, but pointless, to put a \verb+currently+ field in a
\verb+var+ one: the outer \verb+var+ one will prevent from modifying
the inner one, thus making it strictly less usable than a \verb+var+
field (no type modification and no usable copy).

~

Equipped with these rules, we can now type assignments. But as an
intermediate step, we'll spell the simpler rule for the special case
where both left-hand side and right-hand side sequences have only one
element:

$$
\frac{
\begin{array}{c}
\Gamma\vdash E: \T E
\qquad
\T F \rhd \J\T E
\qquad
\Gamma\vdash \update{L}{\T F} = [\sigma]
\qquad
\Gamma[\sigma]\vdash \bar S:\T S
\end{array}
}{
\Gamma\vdash (L\ \#\T F=E; \bar S) : \T S
}
$$

To paraphrase, it must be possible (1) $E$ must be well typed; (2)
this type must be legal to store in a field of $L$'s new type $\T F$;
(3) it must be legal to substitute $L$'s former type with the new one
$\T F$; (4) the rest of the sequence $\bar S$ must be typable with
the type substitution $[\sigma]$ applied in $\Gamma$.

The actual rule, although more intimidating because it deals with
sequences of possibly different lengths, is not more
sophisticated. The two upper premices define a type $\bar\T E$
corresponding to $\T E$ above, with some \Nil-padding if needed to
match the right-hand-side's length. The lower one involving $\rhd$ and
update() corresponds to premices (2) and (3), applied to each (left,
right) pair; and the final premice chains all substitutions together
to type the following statements:

$$
\frac{
  \begin{array}{l}
    (\forall n\in [1...p])\ \Gamma\vdash E_n: \T E_n\quad
    (\forall n\in [p+1...m])\ \T E_n=\Nil\\
    (\forall n\in [1...m])\ 
    \T F_n \rhd \J\T E_n \textrm{ and }
    \Gamma\vdash \update{L_n}{\T F_n} = [\sigma_n]
  \end{array}
  %
  \Gamma[\sigma_1...\sigma_n] \vdash \bar S:\T S
}{
\Gamma\vdash ((L_n\ \#\T F_n)^{\forall n\in[1...m]}=(E_n)^{\forall n\in[1...p]}; \bar S) : \T S
}
$$
\vspace{-2em}
\begin{flushright}
  $\eqlabel{TR-Assign}$
\end{flushright}

\todo{Substitution conflicts within an assignment aren't handled,
  e.g. {\tt x \#[|currently nil] = \{ \}; x.a, x.b = "A", "B"}. The
  simplest solution is to mandate that substitutions have disjoint
  domains within an assignment.}

\paragraph{Functions}
Functions break linearity: they assign arguments to parameters, which
can create a second reference to a table. Moreover, due to Lua's
support for full closures, they capture variables defined outside of
them (these variables are called ``upvalues'' in Lua). Functions don't
use upvalues' content immediately, where the function is defined;
instead, they'll use them whenever they're called, and in between, the
content of a \verb+currently+ field or variable might have been
changed arbitrarily.

For this reason, when typing the function's body, we weaken every
 upvalue through $\rhd$, so that it doesn't rely on any variable
 having a field type of the form $\C\T E$. We elect to type parameters
 as \verb+var+ slots, thus allowing to change their content in the
 function's body. It would have been possible to forbid it by typing
 them with \verb+const+\footnote{I wonder whether typing parameters as
   {\tt currently} inside the function's body would have been
   admissible. It's not trivial: we must not allow to change a field
   in a table, so we must be sure there's no {\tt currently} table
   fields.}.

$$\frac{
  \Gamma_{\textrm{in}} \rhd \Gamma_{\textrm{out}}
  \quad
  \Gamma_{\textrm{in}}[\bar v\leftarrow\V\bar\T E_?]
  \vdash \bar S: \Return{\bar\T E_!}\\
}{
  \Gamma_{\textrm{out}}\vdash
  \Function{\overline{v\ \#\T E_?}}{\bar S}:
  \T E_?\rightarrow\bar\T E_!
}\eqlabel{TR-Function}$$

\noindent
(this rule generalizes $\rhd$ over environments, in the obvious way:
$\Gamma_1\rhd\Gamma_2$ iff $\Gamma_1$ has the same domain $D$ as
$\Gamma_2$, and $(\forall v \in D)\ \Gamma_1(v) \rhd \Gamma_2(v)$)


\paragraph{Function applications}
As we did for \eqlabel{TR-Assign}, let's demystify the function
application rule, by first giving the simplified version with one
parameter, one argument and one result: (1) what's applied must be a
function, (2) the argument must be well typed, and (3) this argument
type must be compatible with the parameter:

$$
\frac{
  \Gamma\vdash E_f: (\T E^?)\rightarrow(\T E^!)
  \qquad
  \Gamma\vdash E:\T E^a 
  \qquad
  \T E^a <: \T E^?
}{
\Gamma\vdash E_f(E^a)_E: (\T E^!)
}
$$

This rule is extended to multiple parameters / arguments / results
almost trivially; the only point to take care of is that the argument
types sequence might be padded with \Nil\ types, if it's shorter than
the parameter types sequence:

$$
\frac{
  \begin{array}{c}
    \Gamma\vdash E_f:
    (\T E^?_n)^{\forall n\in[1...m]}\rightarrow
    (\T E^!_n)^{\forall n\in[1...p]}\\
    (\forall n\in[1...q])\ \Gamma\vdash E^a_n:\T E^a_n
    \quad
    (\forall n\in[q+1...m])\ \T E^a_n = \Nil\\
    (\forall n\in[1...m])\ \T E^a_n <: \T E^?_n
  \end{array}
}{
\Gamma\vdash E_f((E^a_n)^{\forall n\in[1...q]})_E:
(\T E^!_n)^{\forall n\in[1...p]}
}
\quad
\eqlabel{TR-Apply-E}
$$

Finally, function applications in a statement context are typed by
discarding applying \eqlabel{TR-Apply-E}, then forgetting the results
and replacing them with the non-returning-statement type
$\varnothing_{\T S}$:

$$
\frac{
\Gamma\vdash E_f(\bar E)_E: \bar\T E
}{
\Gamma\vdash E_f(\bar E)_S: \varnothing_{\T S}
}
\quad
\eqlabel{TR-Apply-S}
$$

%-*-mode:latex; eval:(whizzytex-mode);-*-
%; whizzy-master lucal.tex

\section{Future work}
This section sums up a list tasks remaining to be completed, in order
to turn this calculus into a useful type-checker for actual Lua
programs.

\begin{itemize}
\item Implement a type checker for the existing calculus.
\item Produce a soundness proof (formal demonstration that a typed
  term can't cause a runtime type error).
\item Combine with gradual typing.
\item Define a pragmatic syntax: allow to omit easily guessed type
  annotations, have sensible defaults for missing indications such as
  field type modifiers, etc.
\item Consider an alternative, Lua compatible syntax for types which
  fits into Lua comments. This version would be backward-compatible
  with plain Lua compilers, and would certainly present similarities
  with Luadoc-like tools.
\item Support for (limited) type inference. Unwritten types must not
  be all interpreted as dynamic types; otherwise, no type checking at
  all would occur in programs which aren't fully annotated. A
  reasonable compromize would probably be that unannotated function
  parameters are dynamic, but unnanotated locals are to be guessed
  through inference.
\item Clarify what typed programs look like. Not all Lua programming
  styles will be supported: for instance, modules which pollute global
  variables will probably not be accepted. The choices about what's
  acceptable for the type system must not only be made: their
  rationales and their price must be carefully justified.
\item Extend the calculus with missing parts of Lua. Some should be
  easy to incorporate in the type system (loop statements,
  pseudo-fields based on \verb+__index+ tables); others
  would have to remain dynamically typed (binary operators,
  varargs...). Here gradual typing really saves the day: constructs
  which no current type system handles satisfactorily, such as
  covariant binary operators, can be left dynamically typed without
  making the whole type system fall apart.
\end{itemize}


%%-*-mode:latex; eval:(whizzytex-mode);-*-

\section{Terms annotation}

In this section, we propose a way to annotate Lua terms with
types. We'll talk about concrete syntax matters here, not only formal
considerations.

First let's define the fully annotated calculus, holding all the
typing annotations potentially needed to typecheck a program. We'll
later discuss the possibility to leave some annotations missing, and
how they should be interpreted. We'll also leave out facilities such
as type alias definitions for now: the goal here is to determine how
much annotation is needed to check a term.

$$
\begin{array}{rcll}
E &::=& L & \eqlabel{E-Left} \\
  &|&   P & \eqlabel{E-Primitive} \\
  &|&   E(\bar E)_E & \eqlabel{E-Apply} \\
  &|&   \Function{\bar A}{\bar S}  & \eqlabel{E-Function} \\
  &|&   \Table{ \compseq{[E^k_n]=E^v_n}{n\in[1...n]} } & \eqlabel{E-Table} \\
L &::=& v\ \sharp \T E & \eqlabel{E-Variable} \\
  &|&   E[E] & \eqlabel{E-Index} \\
P &::=& \langle\textrm{string}\rangle & \eqlabel{E-String} \\
  &|&   \langle\textrm{number}\rangle & \eqlabel{E-Number} \\
  &|&   \texttt{true} & \eqlabel{E-True} \\
  &|&   \texttt{false} & \eqlabel{E-False} \\
  &|&   \texttt{nil} & \eqlabel{E-Nil} \\
  &|&   t & \eqlabel{E-TableRef} \\
  &|&   f & \eqlabel{E-ClosureRef} \\
S &::=& \Local{\bar v} & \eqlabel{S-Local} \\
  &|&   \bar L = \bar E & \eqlabel{S-Assign} \\
  &|&   E(\bar E)_S & \eqlabel{S-Apply} \\
  &|&   \Return \bar E & \eqlabel{S-Return} \\
  &|&   \sharp \Return{\bar\T E}\\
A &::=& v\ |\  v\ \sharp \T E\\
\T E &::=& [\T F; \overline{P:\T F}]\\
  &|&   v\\
  &|&   \star_{\T E}\\
  &|&   \T E[P]\\
  &|&   \bar\T E \rightarrow \bar\T E\\
\bar\T E &::=& \star_{\bar\T E}\ |\ v_{\bar\T E}\ |\ (\overline{ \T E}) \\
\T F &::=& \Tcurrently{\T E}\\
  &|&   \Tvar{\T E}\\
  &|&   \Tconst{\T E}\\
  &|&   \Tfield\\
\end{array}
$$

With this annotation system:
\begin{itemize}
\item function parameters are explicitly typed; The result is
  inferred, but can be forced with a statement annotation;
\item locals aren't typed at decalaration (they contain \Nil\ anyway)),
  but variable assignmnents are annotated. This allows to specify
  variable type changes when allowed by the type system;
\item tables are not annotated, but as soon as they're assigned in a
  variable, this assignement operation is annotated;
\item table field updates. Overriding \Tvar{}fields is OK as long as
  the type doesn't change; for \Tcurrently{}fields, we need to guess
  the type. It probably doesn't matter that much, since those types
  are easily overridden; \todo{a covariant subtyping rule
    might be missing here, ot go from e.g. {\tt currently positive} to
    {\tt var number}}
\item function applications are typed are deduced from the function's
  type;
\item of course $P$ values are typed;
\item table field accesses are only typed if the table is typed;
\item return statements are inferred.
\end{itemize}


Unannotated function parameters will be treated as dynamic types;
we'll try to guess a static type for unannotated local vars, though:
if we don't, all function calls using them will become dynamic, and
little actual type-checking will be able to take place.


%% \begin{displaymath}
%%   \xymatrix{
%%         A \ar[d] \ar[dr] \ar[r] & B \\
%%         D                       & C }
%% \end{displaymath}

\end{document}
