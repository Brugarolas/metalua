\section{Data structures}

\subsection{Algebraic Datatypes (ADT)}

(ADT is also the usual accronym for Abstract DataType. However, I'll
never talk about abstract datatypes in this manual, so there's no
reason to get confused about it. ADT always refers to algebraic
datatypes).

Metalua's distinctive feature is its ability to easily work on program
source codes as trees, and this include a proper syntax for tree
manipulation. The generic table structure offered by Lua is
definitely good enough to represent trees, but since we're going to
manipulate them a lot, we give them a specific syntax which makes them
easier to read and write.

So, a tree is basically a node, with:
\begin{itemize}
\item a tag (a string, stored in the table field named ``{\tt tag}'')
\item some children, which are either sub-trees, or atomic values
  (generally strings, numbers or booleans). These children are stored
  in the array-part\footnote{Tables in Lua can be indexed by integers,
    as regular arrays, or by any other Lua data. Moreover, their
    internal representation is able to optimize both array-style and
    hashtable-style usage, and both kinds of keys can be used in the
    same table. In this manual, I'll refer to the integer-indexed part
    of a table as its array-part, and the other one as its hash-part.}
  of the table, i.e. with consecutive integers as keys.
\end{itemize}

\paragraph{Example 1}

The most canonical example of ADT is probably the inductive list. Such
a list is described either as the empty list \verb+Nil+, or a pair
(called a \verb+cons+ in Lisp) of the first element on one side
(\verb+car+ in Lisp), and the list of remaining elements on the other
side (\verb+cdr+ in Lisp). These will be represented in Lua as
\verb+{ tag = "Nil" }+ and {\tt\{ tag = "Cons", car, cdr \}}. The list
(1, 2, 3) will be represented as:

\begin{verbatim}
{ tag="Cons", 1,
  { tag="Cons", 2,
    { tag="Cons", 3,
      { tag="Nil" } } } }
\end{verbatim}

\paragraph{Example 2}

Here is a more programming language oriented example: imagine that we
are working on a symbolic calculator. We will have to work this:
\begin{itemize}
\item literal numbers, represented as integers;
\item symbolic variables, represented by the string of their
  symbol;
\item formulae, i.e. numbers, variables an/or sub-formulae
  combined by operators. Such a formula is represented by the symbol
  of its operator, and the sub-formulae / numbers / variables it
  operates on.
\end{itemize}
Most operations, e.g. evaluation or simplification, will do different
things depending on whether it is applied on a number, a variable or a
formula. Moreover, the meaning of the fields in data structures depends
on that data type. The datatype is given by the name put in the
\verb+tag+ field. In this example, \verb+tag+ can be one of
\verb+Number+, \verb+Var+ or \verb+Formula+. The formula $e^{i\pi}+1$
would be encoded as:
\begin{verbatim}
{ tag="Formula", "Addition",
  { tag="Formula", "Exponent",
    { tag="Variable", "e" },
    { tag="Formula", "Multiplication",
      { tag="Variable", "i" },
      { tag="Variable", "pi" } } },
  { tag="Number", 1 } }
\end{verbatim}

\paragraph{Syntax}

The simple data above already has a quite ugly representation, so here
are the syntax extensions we provide to represent trees in a more
readable way:

\begin{itemize}
\item The tag can be put in front of the table, prefixed with a
  backquote. For instance, {\tt\{ tag = "Cons", car, cdr \}} can be
  abbreviated as {\tt`Cons\{ car, cdr \}}.
\item If the table contains nothing but a tag, the braces can be
  omitted. Therefore, \verb+{ tag = "Nil" }+ can be abbreviated as
  \verb+`Nil+ (although \verb|`Nil{ }| is also legal).
\item If there is only one element in the table besides the tag, and
  this element is a literal number or a literal string, braces can be
  omitted. Therefore {\tt\{ tag = "Foo", "Bar" \}} can be abbreviated
  as \verb+`Foo "bar"+.
\end{itemize}

With this syntax sugar, the $e^{i\pi}+1$ example above would read:
\begin{verbatim}
`Formula{ "Addition",
   `Formula"{ "Exponent",
      `Variable "e",
      `Formula{ "Multiplication",
                `Variable "i",
                `Variable "pi" } },
   `Number 1 }
\end{verbatim}

Notice that this is a valid description of some tree structure in metalua, but
it's not a representation of metalua code: metalua code is represented as tree
structures indeed, but a structure different from this example's one. In other
words, this is an ADT, but not an AST.

For the record, the metalua (AST) representation of the code {\tt"1+e\^\ (i*pi)"}
is:
\begin{verbatim}
`Op{ "add", `Number 1,
     `Op{ "pow", `Id "e",
          `Op{ "mul", `Id "i", `Id "pi" } } }
\end{verbatim}

After reading more about AST definition and manipulation tools, you'll hopefully
be convinced that the latter representation is more powerful.

\subsection{Abstract Syntax Trees (AST)}

An AST is an Abstract Syntax Tree, a data representation of source
code suitable for easy manipulation. AST are just a particular usage
of ADT, and we will represent them with the ADT syntax described
above.

\paragraph{Example} this is the tree representing the source code
\verb+print(foo, "bar")+:

\verb+`Call{ `Id "print", `Id "foo", `String "bar" }+

Metalua tries, as much as possible, to shield users from direct AST
manipulation, and a thorough knowledge of them is generally not
needed. Metaprogrammers should know their general form, but it is
reasonnable to rely on a cheat-sheet to remember the exact details of
AST structures. Such a summary is provided
in an appendix of this tutorial, as a reference when dealing with them.

In the rest of this section, we will present the translation from Lua
source to their corresponding AST.

\subsection{AST  $\Longleftrightarrow$ Lua source translation}

This subsection explains how to translate a piece of lua source code
into the corresponding AST, and conversely. Most of time, users will
rely on a mechanism called quasi-quotes to produce the AST they will
work with, but it is sometimes necessary to directly deal with AST,
and therefore to have at least a superficial knowledge of their
structure.

\subsubsection{Expressions}

The expressions are pieces of Lua code which can be evaluated to give a
value. This includes constants, variable identifiers, table
constructors, expressions based on unary or binary operators, function
definitions, function calls, method invocations, and index selection
from a table.

Expressions should not be confused with statements: an expression has
a value with can be returned through evaluation, whereas statements
just execute themselves and change the computer state (mainly memory
and IO). For instance, \verb|2+2| is an expression which evaluates to
4, but \verb|four=2+2| is a statement, which sets the value of
variable \verb|four| but has no value itself.

\paragraph{Number constants}
A number is represented by an AST with tag \verb+Number+ and the
number value as its sole child. For instance, \verb+6+ is represented
by \verb+`Number 6+\footnote{As explained in the section about ADT,
  {\tt `Number 6} is exactly the same as {\tt `Number\{ 6 \}}, or
  plain Lua {\tt\{ tag="Number", 6\}} }.

\paragraph{String constants}
A string is represented by an AST with tag \verb+String+ and the
string as its sole child. For instance, \verb+"foobar"+ is
represented by:\\
\verb+`String "foobar"+.

\paragraph{Variable names}
A variable identifier is represented by an AST with tag \verb+Id+ and the
number value as its sole child. For instance, variable \verb+foobar+ is
represented by \verb+`Id "foobar"+.

\paragraph{Other atomic values}
Here are the translations of other keyword-based atomic values:
\begin{itemize}
\item \verb+nil+ is encoded as \verb+`Nil+\footnote{which is a
  short-hand for {\tt`Nil\{ \}}, or {\tt\{ tag="Nil" \}} in plain Lua.};
\item \verb+false+ is encoded as \verb+`False+;
\item \verb+true+ is encoded as \verb+`True+;
\item \verb+...+ is encoded as \verb+`Dots+.
\end{itemize}

\paragraph{Table constructors}
A table constructor is encoded as:

\verb+`Table{ ( `Pair{ expr expr } | expr )* }+

This is a list, tagged with \verb+Table+, whose elements are either:
\begin{itemize}
\item the AST of an expression, for array-part entries without an
  explicit associated key;
\item a pair of expression AST, tagged with \verb+Pair+: the first
  expression AST represents a key, and the second represents the value
  associated to this key.
\end{itemize}

\subparagraph{Examples}
\begin{itemize}

\item The empty table \verb+{ }+ is represented as \verb+`Table{ }+;

\item \verb+{1, 2, "a"}+ is represented as:\\
  \verb+`Table{ `Number 1, `Number 2, `String "a" }+;

\item \verb+{x=1, y=2}+ is syntax sugar for \verb+{["x"]=1, ["y"]=2}+,
  and is represented by {\tt`Table\{ `Pair\{ `String "x", `Number 1
    \}, `Pair\{ `String "y", `Number 2\} \}};

\item indexed and non-indexed entries can be mixed:
  \verb+{ 1, [100]="foo", 3}+ is represented as {\tt`Table\{ `Number
    1, `Pair\{ `Number 100, `String "foo"\}, `Number 3 \}};

\end{itemize}

\paragraph{Binary Operators}
Binary operations are represented by {\tt`Op\{ operator, left,
  right\}}, where \verb+operator+ is a the operator's name as one of
the strings below, \verb+left+ is the AST of the left operand, and
\verb+right+ the AST of the right operand.

The following table associates a Lua operator to its AST name:

\begin{center}
\begin{tabular}{|c|c||c|c||c|c||c|c|}
  \hline
  \bf Op. & \bf AST &
  \bf Op. & \bf AST &
  \bf Op. & \bf AST &
  \bf Op. & \bf AST \\

  \hline\hline %%%%%%%%%%%%%%%%%
  \verb|+|   & \verb+"add"+    &
  \verb+-+   & \verb+"sub"+    &
  \verb+*+   & \verb+"mul"+    &
  \verb+/+   & \verb+"div"+    \\
  \hline %%%%%%%%%%%%%%%%%%%%%%%
  \verb+%+   & \verb+"mod"+    &
  \verb+^+   & \verb+"pow"+    &
  \verb+..+  & \verb+"concat"+ &
  \verb+==+  & \verb+"eq"+     \\
  \hline %%%%%%%%%%%%%%%%%%%%%%%
  \verb+<+   & \verb+"lt"+     &
  \verb+<=+  & \verb+"le"+     &
  \verb+and+ & \verb+"and"+    &
  \verb+or+  & \verb+"or"+     \\
  \hline %%%%%%%%%%%%%%%%%%%%%%%
\end{tabular}
\end{center}

Operator names are the same as the corresponding Lua metatable entry,
without the prefix {\tt"\_\,\_"}. There are no operators for
\verb+~=+, \verb+>=+ and \verb+>+: they can be simulated by swapping
the arguments of \verb+<=+ and \verb+<+, or adding a \verb+not+ to
operator \verb+==+.

\subparagraph{Examples}
\begin{itemize}
\item \verb|2+2| is represented as
  \verb|`Op{ 'add', `Number 2, `Number 2 }|;
\item \verb|1+2*3| is represented as:\\[-2em]
\begin{verbatim}
`Op{ 'add', `Number 1,
     `Op{ 'mul', `Number 2, `Number 3 } }
\end{verbatim}
\item \verb|(1+2)*3| is represented as:\\[-2em]
\begin{verbatim}
`Op{ 'mul, `Op{ 'add', `Number 1, `Number 2 },
     `Number 3 } }
\end{verbatim}

  \verb|`Op{ 'mul', `Op{ 'add', `Number 1, `Number 2 }, `Number 3 }|
\item \verb|x>=1 and x<42 | is represented as:\\[-2em]
\begin{verbatim}
`Op{ 'and', `Op{ 'le', `Number  1, `Id "x" },
            `Op{ 'lt', `Id "x", `Number 42 } }

\end{verbatim}
\end{itemize}

\paragraph{Unary Operators}
Unary operations are similar to binary operators, except that they
only take the AST of one subexression. The following table associates
a Lua unary operator to its AST:

\begin{center}
\begin{tabular}{|c|c||c|c||c|c|}
  \hline
  \bf Op. & \bf AST &
  \bf Op. & \bf AST &
  \bf Op. & \bf AST \\

  \hline\hline %%%%%%%%%%%%%%
  \verb+-+   & \verb+"unm"+ &
  \verb+#+   & \verb+"len"+ &
  \verb+not+ & \verb+"not"+ \\
  \hline %%%%%%%%%%%%%%%%%%%%
\end{tabular}
\end{center}

\subparagraph{Examples}
\begin{itemize}
\item \verb|-x| is represented as \verb|`Op{ 'unm', `Id "x" }|;
\item \verb|-(1+2)| is represented as:\\
  \verb|`Op{ 'unm', `Op{ 'add', `Number 1, `Number 2 } }|
\item \verb|#x| is represented as
  \verb|`Op{ 'len', `Id "x" }|
\end{itemize}

\paragraph{Indexed accesses}
They are represented by an AST with tag \verb+Index+, the table's AST
as first child, and the key's AST as second child.

\subparagraph{Examples}
\begin{itemize}
\item \verb+x[3]+ is represented as \verb+`Index{ `Id "x", `Number 3 }+;
\item \verb+x[3][5]+ is represented as:\\
  \verb+`Index{ `Index{ `Id "x", `Number 3 }, `Number 5 }+
\item \verb+x.y+ is syntax sugar for \verb+x["y"]+, and is represented as:\\
  \verb+`Index{ `Id "x", `String "y" }+
\end{itemize}

Notice that an index AST can also appear as the left-hand side of
an assignment, as shall be shown in the subsection dedicated to
statements.

\paragraph{Function calls}
A function call AST has the tag \verb+Call+, the called function's AST
as the first child, and its arguments as the remaining children.

\subparagraph{Examples}
\begin{itemize}
\item \verb+f()+ is represented as \verb+`Call{ `Id "f" }+;
\item \verb+f(x, 1)+ is represented as
  \verb+`Call{ `Id "f", `Id "x", `Number 1 }+;
\item \verb+f(x, ...)+ is represented as
  \verb+`Call{ `Id "f", `Id "x", `Dots }+.
\end{itemize}

Notice that function calls can be used as expressions, but also as statements.

\paragraph{Method invocations}
A method invocation AST has the tag \verb+Invoke+, the object's AST as
the first child, the string name of the method as the second child, and
the arguments as remaining children.

\subparagraph{Examples}
\begin{itemize}
\item \verb+o:f()+ is represented as \verb+`Invoke{ `Id "o", String "f" }+;
\item \verb+o:f(x, 1)+ is represented as:\\
  \verb+`Invoke{ `Id "o", `String "f", `Id "x", `Number 1 }+;
\item \verb+o:f(x, ...)+ is represented as:\\
  \verb+`Invoke{ `Id "o", `String "f", `Id "x", `Dots }+;
\end{itemize}

Notice that method invocations can be used as expressions, but also as
statements.  Notice also that ``{\tt function o:m (x) return x end}'' is
not a method invocation, but syntax sugar for the statement ``{\tt
o["f"] = function (self, x) return x end}''. See the paragraph about
assignment in the statements subsection for its AST representation.


\paragraph{Function definitions}
A function definition consists of a list of parameters and a block of
statements. The parameter list, which can be empty, contains only
variable names, represented by their \verb+`Id{...}+ AST, except for
the last element of the list, which can also be a dots AST \verb+`Dots+
(to indicate that the function is a vararg function).

The block is a list of statement AST, optionally terminated with a
\verb+`Return{...}+ or \verb+`Break+ pseudo-statement. These
pseudo-statements will be described in the statements subsection.

FIXME: finally, return and break will be considered as regular
statements: it's useful for many macros.

The function definition is encoded as
\verb+`Function{ parameters block }+

\subparagraph{Examples}
\begin{itemize}

\item \verb+function (x) return x end+ is represented as:\\
  \verb+`Function{ { `Id x } { `Return{ `Id "x" } } }+;

\item \verb+function (x, y) foo(x); bar(y) end+ is represented as:
\begin{verbatim}
`Function{ { `Id x, `Id y }
           { `Call{ `Id "foo", `Id "x" },
             `Call{ `Id "bar", `Id "y" } } }
\end{verbatim}

\item \verb+function (fmt, ...) print (string.format (fmt, ...)) end+
  is represented as:
\begin{verbatim}
`Function{ { `Id "fmt", `Dots }
           { `Call{ `Id "print",
                    `Call{ `Index{ `Id "string",
                                   `String "format" },
                           `Id "fmt",
                           `Dots } } } }
\end{verbatim}

\item \verb+function f (x) return x end+ is not an expression, but a
  statement: it is actually syntax sugar for the assignment {\tt f =
    function (x) return x end}, and as such, is represented as:
\begin{verbatim}
`Let{ { `Id "f" },
      { `Function{ {`Id 'x'} {`Return{`Id 'x'} } } } }
\end{verbatim}
  (see assignment in the statements subsection for more details);

\end{itemize}

\paragraph{Parentheses}

In Lua, parentheses are sometimes semantically meaningful: when the
parenthesised expression returns multiple values, putting it between
parentheses forces it to return only one value. For instance, ``{\tt
  local function f() return 1, 2, 3 end; return \{ f() \}}'' will
return ``{\tt\{1, 2, 3\}}'', whereas ``{\tt local function f() return
  1, 2, 3 end; return \{ (f()) \}}'' will return ``{\tt\{ 1 \}}''
(notice the parentheses around the function call).

Parentheses are represented in the AST as a node ``{\tt`Paren\{
  \}}''. The second example above has the following AST:

\begin{verbatim}
{ `Localrec{ { `Id "f" },
             { `Function{ { },
                          `Return{ `Number 1,
                                   `Number 2,
                                   `Number 3 } } } },
  `Return{ `Table{ `Paren{ `Call{ `Id "f" } } } } }
\end{verbatim}

\subsubsection{Statements}

Statements are instructions which modify the state of the
computer. There are simple statements, such as variable assignments,
local variable declarations, function calls and method invocations;
there are also control structure statements, which take simpler
statements and modify their action: these are if/then/else,
repeat/until, while/do/end, for/do/end and do/end statements.

\paragraph{Assignments}
A variable assignment \verb+a, b, c = foo, bar+ is represented by the
AST \verb+`Set{ lhs, rhs }+, with {\tt lhs} being a list of variables or
table indexes, and {\tt rhs} the list of values assigned to them.

\subparagraph{Examples}
\begin{itemize}

\item \verb+x[1]=2+ is represented as:\\
  \verb+`Set{ { `Index{ `Id "x", `Number 1 } }, { `Number 2 } }+;

\item \verb+a, b = 1, 2+ is represented as:\\
  \verb+`Set{ { `Id "a",`Id "b" }, { `Number 1, `Number 2 } }+;

\item \verb+a = 1, 2, 3+ is represented as:\\
  \verb+`Set{ { `Id "a" }, { `Number 1, `Number 2, `Number 3 } }+;

\item \verb+function f(x) return x end+ is syntactic sugar for:\\
  \verb+f = function (x) return x end+. As such, it is represented as:\\[-2em]
\begin{verbatim}
`Set{ { `Id "f" },
      { `Function{ {`Id 'x'}  {`Return{ `Id "x" } } } } }
\end{verbatim}

\item \verb+function o:m(x) return x end+ is syntactic sugar for:\\
  \verb+o["f"] = function (self, x) return x end+, and as such, is
  represented as:\\[-2em]
\begin{verbatim}
`Set{ { `Index{ `Id "o", `String "f" } },
      { `Function{ { `Id "self, "`Id x }
                   { `Return{ `Id "x" } } } } }
\end{verbatim}

\end{itemize}

\paragraph{Local declarations}
A local declaration \verb+local a, b, c = foo, bar+ works just as an
assignment, except that the tag is \verb+Local+, and it is allowed to
have an empty list as values.

\subparagraph{Examples}
\begin{itemize}

\item \verb+local x=2+ is represented as:\\
  \verb+`Local{ { `Id "x" }, { `Number 2 } }+;

\item \verb+local a, b+ is represented as:\\
  \verb+`Local{ { `Id "a",`Id "b" }, { } }+;

\end{itemize}

\paragraph{Recursive local declarations}
In a local declaration, the scope of local variables starts {\em
  after} the statement. Therefore, it is not possible to refer to a
variable inside the value it receives, and
``{\tt local function f(x) f(x) end}'' is not equivalent to
``{\tt local f = function (x) f(x) end}'': in the latter, the \verb|f|
call inside the function definition probably refers to some global
variable, whereas in the former, it refers to the local variable
currently being defined (f is therefore a function looping forever).

To handle this, the AST syntax defines a special \verb|`Localrec|
local declaration statement, in which the variables enter into scope
{\em before} their content is evaluated. Therefore, the AST
corresponding to {\tt local function f(x) f(x) end} is:
\begin{verbatim}
`Localrec{ { `Id "f" },
           { `Function{ { `Id x }
                        { `Call{ `Id "f", `Id "x" } } } } }
\end{verbatim}

\caveat{In the current implementation, both variable name lists and
  value lists have to be of lenght 1. This is enough to represent
  {\tt local function ... end}, but should be generalized in the
  final version of Metalua.}


\paragraph{Function calls and method invocations}
They are represented the same way as their expression counterparts,
see the subsection above for details.

\paragraph{Blocks and pseudo-statements}
Control statements generally take a block of instructions as
parameters, e.g. as the body of a \verb|for| loop. Such statement
blocks are represented as the list of the instructions they
contain. As a list, the block itself has no \verb|tag| field.

\subparagraph{Example}
\verb|foo(x); bar(y); return x,y| is represented as:
\begin{verbatim}
{ `Call{ `Id "foo", `Id "x" },
  `Call{ `Id "bar", `Id "y" },
  `Return{ `Id "x", `Id "y" } }
\end{verbatim}

\paragraph{Do statements}
These represent \verb|do ... end| statements, which limit the scope of
local variables. They are represented as blocks with a \verb|Do| tag.

\subparagraph{Example}
\verb|do foo(x); bar(y); return x,y end| is represented as:
\begin{verbatim}
`Do{ `Call{ `Id "foo", `Id "x" },
     `Call{ `Id "bar", `Id "y" },
     `Return{ `Id "x", `Id "y" } }
\end{verbatim}

\paragraph{While statements}
\verb|while <foo> do <bar1>; <bar2>; ... end| is represented as \\
\verb|`While{ <foo>, { <bar1>, <bar2>, ... } }|.

\paragraph{Repeat statements}
\verb|repeat <bar1>; <bar2>; ... until <foo>| is represented as \\
\verb|`Repeat{ { <bar1>, <bar2>, ... }, <foo> }|.

\paragraph{For statements}

{\tt for x=<first>,<last>,<step> do <foo>; <bar>; ... end} is
represented as {\tt `Fornum\{ `Id "x", <first>, <last>, <step>, \{
  <foo>, <bar>, ... \} \}}.

The \verb|step| parameter can be omitted if it is equal to 1.

\begin{verbatim}
for x1, x2... in e1, e2... do
  <foo>;
  <bar>;
  ...
end
\end{verbatim}
is represented as:\\
{\tt `Forin\{ \{`Id "x1",`Id "x2",...\}, \{ <e1>, <e2>,... \} \{
 <foo>, <bar>, ... \} \}}.

\paragraph{If statements}
``If'' statements are composed of a series of (condition, block)
pairs, and optionally of a last default ``else'' block. The conditions
and blocks are simply listed in an \verb|`If{ ... }| ADT. Notice that
an ``if'' statement without a final ``else'' block will have an even
number of children, whereas a statement with a final ``else'' block
will have an odd number of children.

\subparagraph{Examples}
\begin{itemize}

\item \verb+if <foo> then <bar>; <baz> end+ is represented as:\\
  \verb+`If{ <foo>, { <bar>, <baz> } }+;

\item \verb+if <foo> then <bar1> else <bar2>; <baz2> end+
  is represented as:
  \verb+`If{ <foo>, { <bar1> }, { <bar2>, <baz2> } }+;

\item
  \verb+if <foo1> then <bar1>; <baz1> elseif <foo2> then <bar2>; <baz2> end+
  \\ is represented as: \\
  \verb+`If{ <foo1>, { <bar1>, <baz1> }, <foo2>,{ <bar2>, <baz2> } }+;

\item
\begin{verbatim}
if     <foo1> then <bar1>; <baz1>
elseif <foo2> then <bar2>; <baz2>
else               <bar3>; <baz3> end+
\end{verbatim}
is represented as:
\begin{verbatim}
`If{ <foo1>, { <bar1>, <baz1> },
     <foo2>, { <bar2>, <baz2> },
             { <bar3>, <baz3> } }
\end{verbatim}

\end{itemize}

\paragraph{Breaks and returns}
Breaks are represented by the childless \verb|`Break| AST. Returns are
retpresented by the (possibly empty) list of returned values.

\subparagraph{Example}
{\tt return 1, 2, 3} is represented as:\\
{\tt`Return\{ `Number 1, `Number 2, `Number 3 \}}.

\subsubsection{Extensions with no syntax}

A couple of AST nodes do not exist in Lua, nor in Metalua native
syntax, but are provided because they are particularly useful for
writing macros. They are presented here.

\paragraph{Goto and Labels}

Labels can be string AST, identifier AST, or simply string; they
indicate a target for goto statements.  A very common idiom is ``{\tt
  local x = mlp.gensym(); ... `Label\{ x \} }''. You just jump to that
label with ``{\tt `Goto\{ x \} }''.

Identifiers, string AST or plain strings are equivalent:
``{\tt`Label\{ `Id "foo"\}}'' is synonymous for ``{\tt`Label\{ `String
  "foo"\}}'' and ``{\tt`Label "foo"}''. The same equivalences apply
to gotos, of course.

Labels are local to a function; you can safely jump out of a block,
but if you jump {\em into} a block, you're likely to get into unspecified
trouble, as local variables will be in a random state.

\paragraph{Statements in expressions}
A common need when writing a macro is to insert a statement in the
middle of an expression. It can be done by using an anonymous function
closure, but that would be expensive, so Metalua offers a better
solution. The \verb|`Stat| node evaluates a statement block in the
middle of an expression, then returns an arbitrary expression as its
result. Notice one important point: the expression is evaluated in
the block's context, i.e. if there are some local variables declared
in the block, the expression can use them.

For instance, {\tt `Stat\{ +\{local x=3\}, +\{x\}\}} evaluates to 3.

%\subsubsection{Formal translation defintion}

%\caveat{FIXME: here should go a formal, inductive definition of
%  AST/syntax translation, to serve as a reference}

\subsection{AST/source mapping with {\tt lineinfo}}

It is sometimes necessary to retrieve, from an AST, information which
existed in the source it comes from, but which doesn't appear in the
AST itself because it has no semantic value: comments, presentation
details such as spacing or use of syntax sugar... To allow this, all
AST nodes generated by Metalua---or, more accurately, through the {\tt
gg} library--- have a {\tt lineinfo} attribute which describe how the
map to the original sources. This section describes how {\tt
  lineinfo}s work.

\paragraph{lineinfo}
Each node generated from a parser comes from a substring of a source
file. for instance, if you compile {\tt src="a(); b()"}, the node
{\tt node = `Call\{ `Id "b" \}} representing the call to function {\tt b}
comes from the substring at offsets 6...8. As such, these two
positions 6 and 8 will be stored as {\tt node.lineinfo.first} and {\tt
  node.lineinfo.last} respectively. Every node coming from a {\tt gg}
parser will have these positions {\tt lineinfo.first} and {\tt
  lineinfo.last}.

{\tt lineinfo} nodes have a dedicated \verb|__tostring| method, so
that they don't mess up the display of an AST more than necessary.

\paragraph{position}
As we've seen above, the main purpose of a {\tt lineinfo} is to store
a pair of positions. However, those positions aren't represented as
simple numbers: they are structures with fields {\tt line}, {\tt
column} and {\tt offset} which respectively describe the position as a
line number (starting from one), an offset within the line, and the
offset from the beginning of the string. They also have a {\tt source}
field which stores the name of the source file it comes from, if
available.

In addition to this offset information, positions store information
about the comments which were before and after the node in the source
respectively. If a node {\tt n} was preceded by some comments, then
{\tt n.lineinfo.first.comments} contain information about these
comments. Similarly, if {\tt n} was followed by some comments,
information about these can be retrieved as\\
{\tt n.lineinfo.last.comments}. 

\paragraph{comments}
There can be several comments preceding of following a node, and
represented by a {\tt comments} field. This field has itself a {\tt
lineinfo} field with two positions, which spans from the beginning of
the first comment to the end of the last comment. It also has, in its
array part, a list of each individual comment. These individual
comments can therefore be retrieved as \verb|comments[1]|,
\verb|comments[2]| etc. Each individual comment also has its own {\tt
  lineinfo}. It has, under index 1, its text, without the \verb|--| or
\verb|--[==[...]==]| markers.

Several consecutive ``\verb|--|'' short comment lines are counted as a
single comment. For instance, \verb|"-- x\n-- y\n"| is a single
comment whose text is \verb|" x\n y"|. However,
\verb|"-- x\n\n-- y\n"| are two separate comments, because of the
empty line separating them.




