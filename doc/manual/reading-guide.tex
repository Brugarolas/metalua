\section*{Reading guide}
This manual tries to be as comprehensive as possible; however, you don't
necessarily have to read all of it before starting to do interesting stuff with
metalua. Here's a brief summary of what the different parts of the manual
address, and why you might want to read them immediately---or not.

\begin{itemize}
\item{\bf Before reading this manual:} metalua is based on Lua, so
  you'll need a minimal level of Lua proficiency. You can probably get
  away without knowing much about metatables, environments or
  coroutines, but you need to be at ease with basic flow control,
  scoping rules, first-class functions, and the whole
  everything-is-a-table approach.
\item{\bf Meta-programming in metalua:} this chapter exposes the generic principles
  of static meta-programming: meta-levels in sources, AST representation of
  code, meta-operators. You need to read this carefully if you plan to write any
  non-trivial meta-programming code and you've never used languages like Common
  Lisp, camlp4 or Converge. If you're familiar with one of these, a cursory look
  over this chapter might be enough for you.
\item{\bf Standard meta-programming libraries:} these are the tools that will allow
  you to effectively manipulate code; the more advanced an extension you want to
  write the more of these you'll want to know.
  \begin{itemize}
  \item{\bf mlp} is the dynamically extensible metalua parser. You need to know it
    if you want to change or extend the language's syntax
  \item{\bf gg} is the grammar generator, the library which lets you manipulate
    dynamic parsers. You need to know it in order to do anything useful with
    mlp.
  \item{\bf match} is an extension supporting structural pattern matching (which has
    almost nothing to do with regular expressions on strings). It's a construct
    taken from the ML language familly, which lets you manipulate advanced data
    structures in very powerful ways. It's extremely helpful, among others, when
    working with AST, i.e. for most interesting meta-programs.
  \item{\bf walk} is a code walker generator: something akin to a visitor pattern,
    which will help you to write code analysers or transformers. Whenever you
    want to find and transform all return statements in an AST, rename some
    conflicting local variables, check for the presence of nested for loops
    etc., you'll have to write a code walker, and walk will get you there much
    faster. 
  \item{\bf hygiene} offers hygienic macros, i.e. protects you from accidental
    variable captures. As opposed to e.g. Scheme, macro writing is not limited
    to a term rewriting system in metalua, which gives more power to the
    programmer, but prohibits completely automating macro hygienization. If
    you wrote an extension and you wanted to raise it to production-quality,
    you'd need among others to protect its users from variable captures, and
    you'd need to hygienize it. If you don't feel like cluttering your code
    with dozens of {\tt gensym} calls, you'll want to use the macro hygienizer.
  \item{\bf dollar:} if you wrote a macro, but don't feel the need to give it a
    dedicated syntax extension, this library will let you call the macro as a
    regular function call, except that it will be prefixed with a ``{\tt\$}''.
  \end{itemize}
  \item{\bf General purpose libraries:} Lua strives at staying minimalist, and does
    not come with batteries included; you're expected to grab them separately,
    currently from luaforge, and eventually from a Lua Rocks repository. Metalua
    needs quite some support to run, and relies on a number of imported and
    custom-built libraries. Most of them can be reused for many other purposes
    including yours.\\
    A whole category of metalua users who want to use third party libraries,
    rather than reinventing their own wheels, will be primarily interested in
    these.
    \begin{itemize}
    \item{\bf metalua.runtime:} extensions to Lua core libraries: base, table,
      string.
    \item{\bf metalua.compiler:} mlc offers a consistent interface to metalua
      compilation and code representation transformers. 'package', 'loadstring',
      'dostring', 'loadfile' and 'dofile' are also updated to handle metalua
      source files.
    \item{\bf clopts} simplifies and unifies the handling of command line options
      for metalua programs.
    \item{\bf springs} brings together Lua Ring's handling of separate Lua universes
      with Pluto's communication capabilities.
    \item{\bf clist} offers an extended tables-as-lists interface: lists by
      comprehension {\em \`a la} Haskell or Python, list chunks etc.
    \item{\bf xglobal} makes global variables declaration mandatory, for safer
      programming, with almost no runtime overhead, and a syntax consistant with
      local variables declaration.
    \item{\bf anaphoric} introduces anaphoric control structures, akin to Common
      Lisp's {\tt aif}-family macros.
    \item{\bf trycatch} provides a proper exception system, with reliable finally
      blocks and exception catching by structural pattern matching.
    \item{\bf log} eases the terminal logging of variables, mainly for those from
      the Printf-based School of Debugging.
    \item{\bf types} offers dynamic type checking to metalua programs. It supports
      variable typing as opposed to value typing, and advanced type system
      features (polymorphism, dependant types etc.).
    \end{itemize}
  \item{\bf Examples and tutorials}: this chapter lists a series of tiny
    meta-programs whose main purpose is didactic, and walks through the detailed
    implementation of a couple of non-trivial extensions.
\end{itemize}
