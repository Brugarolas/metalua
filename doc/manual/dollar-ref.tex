\section{Library {\tt dollar}: generic syntax for macros}
When you write a short-lived macro which takes reasonably short arguments, you
generally don't want to write a supporting syntax for it. The dollar library
allows you to call it in a generic way: if you store your macro in the table
{\tt mlp.macros}, say as function {\tt mlp.macros.foobar}, then you can call it
in your code as {\tt\$foobar(arg1, arg2...)}: it will receive as parameters the
ASTs of the pseudo-call's arguments.

\paragraph{Example}~
\begin{verbatim}
-{ block:
   require 'dollar'
   function mlp.macros.LOG(id)
      match id with `Id{ id_name } -> 
         printf("%s = %s"), id_name, table.tostring(id)
      end
   end }

local x = { 1, 2, 3 }
$LOG(x) -- prints "x = { 1, 2, 3 }" when executed
\end{verbatim}
